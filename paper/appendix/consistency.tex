
We now provide a result showing that under sufficient regularity conditions, one can interchange the order of differentiation and expectation of functionals of off-parameter resampled particle estimates, showing that if an off-parameter resampled particle estimate is consistent for some estimand, its derivative is consistent for the derivative of the estimand as well.

One may wonder why this may be necessary, given that we have already shown strong consistency for measurable bounded functionals in Theorem \ref{thm:off-parameter-targeting}. If we require the gradient to be bounded as well, then the gradient is also a bounded functional. The answer is that the strong consistency is for expectations under the filtering distribution, so Theorem \ref{thm:off-parameter-targeting} only establishes strong consistency of the gradient of the estimate to its expectation under the filtering distribution, $\nabla_\theta \eta_n^J(h_\theta)  \stackrel{a.s.}{\to} \eta_n(\nabla_\theta h_\theta)$, which is not a-priori the gradient of the estimand $\nabla_\theta \eta_n(h_\theta)$. We require an interchange of the derivative and expectation, which we show is possible when the particle estimates have two continuous uniformly bounded derivatives over all $J$ below.

\begin{thm}[Off-Parameter Particle Filters Yield Strongly Consistent Estimates of Derivatives of Functionals]
    Let $h_\theta : \gX \to \R$ be a measurable bounded functional of particles, where $\eta_n^J(h_\theta)$ has two continuous derivatives uniformly bounded over all $J$ by $H^*$ for almost every $\omega\in\Omega$ and $\theta \in \Theta$. If it holds that $\eta_n^J(h_\theta) \stackrel{a.s.}{\to} \eta(h_\theta) = h^*_\theta$, then we also have that $\nabla_\theta \eta_n^J(h_\theta)  \stackrel{a.s.}{\to} \eta_n(\nabla_\theta h_\theta) = \nabla_\theta \eta_n(h_\theta) = \nabla_\theta h^*_\theta$. 
\end{thm}
\begin{proof}
    Fix $\omega \in \Omega$. The sequence $(\nabla_\theta \eta_n^J(h_\theta)(\omega))_{J \in \mathbb{N}}$ is uniformly bounded over all $J$, by assumption. The sequence is also uniformly equicontinuous. To see this, by assumption, the second derivative of $\eta_n^J(h_\theta)(\omega)|_{\theta=\theta'}$ is also uniformly bounded over all $J$ for almost every $\omega\in \Omega$ and every $\theta' \in \Theta$. A set of functions with derivatives bounded by the same constant is uniformly Lipschitz, and therefore uniformly equicontinuous. So the sequence $(\eta_n^J(h_\theta)\omega))_{J \in \mathbb{N}}$ is uniformly equicontinuous over $\theta$ for almost every $\omega \in \Omega$. 
    Explicitly, for almost every $\omega \in \Omega$ and every $\epsilon>0$, there exists some $\delta(\omega)>0$ such that for every $||\theta - \theta'||_{\infty}<\delta$ and every $J \in \mathbb{N}$ we have that
    $$||\eta_n^J(h_\theta)(\omega)-\eta_n^J(h_{\theta'})(\omega)||_\infty < \epsilon.$$
    Then, by Arzela-Ascoli, there exists a uniformly convergent subsequence. We claim that there is only one subsequential limit. When the gradient is bounded, we can treat the gradient as a bounded functional. So by Theorem \ref{thm:off-parameter-targeting} the sequence $(\eta_n^J(h_\theta)(\omega))_{J \in \mathbb{N}}$ converges pointwise for $\theta=\phi$ and almost every $\omega \in \Omega$, and there is therefore only one subsequential limit. The sequence therefore converges uniformly to its limit $\lim_{J \to \infty} \eta_n^J(h_\theta)(\omega).$ Therefore, with uniform convergence for the derivatives established, we can swap the limit and derivative, and obtain that for almost every $\omega \in \Omega$, 
    $$\lim_{J \to \infty} \eta_n^J(h_\theta)(\omega) = \nabla_\theta \lim_{J \to \infty} \eta_n^J(h_\theta)(\omega).$$
Again from Theorem \ref{thm:off-parameter-targeting}, we know that
    $$\eta_n^J(h_\theta)(\omega) \to \eta_n(h_\theta) = h^*_\theta$$ for almost every $\omega\in\Omega$.
    We then have that for almost every $\omega \in \Omega$, 
    $$\lim_{J \to \infty} \eta_n^J(h_\theta)(\omega) = \nabla_\theta \eta_n(h_\theta) = \nabla_\theta h^*_\theta,$$
    as we wanted. 
    \end{proof}
The proof of Theorem \ref{thm:mop-grad-consistency} is now merely a corollary, where we apply $\eta_n^J (h_\theta) = \hat\lik^1_J(\theta)$, $\eta_n(h_\theta) = \lik(\theta) = h_\theta^*$, and then use the continuous mapping theorem. 



% For completeness, we now provide a longer version of the proof of Theorem \ref{thm:mop-grad-consistency}.
% \begin{proof}
%     Fix $\omega \in \Omega$, and set $\phi = \theta$, where $\theta$ is the point we wish to evaluate the gradient at. The sequence $(\nabla_\theta \hat\lik^1_J(\theta)(\omega))_{J \in \mathbb{N}}$ is uniformly bounded over all $J$, by Assumption \ref{assump:local-bounded-derivative}. The sequence is also uniformly equicontinuous. To see this, by Assumption \ref{assump:local-bounded-derivative}, the second derivative of $\hat\lik^1_J(\theta)(\omega)|_{\theta=\theta'}$ is also bounded by $H^*$ for almost every $\omega\in \Omega$ and every $\theta' \in \Theta$. A set of functions with derivatives bounded by the same constant is uniformly Lipschitz, and therefore uniformly equicontinuous. So the sequence $(\nabla_\theta \hat\lik^1_J(\theta)(\omega))_{J \in \mathbb{N}}$ is uniformly equicontinuous for almost every $\omega \in \Omega$. 
%     Explicitly, for almost every $\omega \in \Omega$ and every $\epsilon>0$, there exists some $\delta(\omega)>0$ such that for every $||\theta - \theta'||_{\infty}<\delta$ and every $J \in \mathbb{N}$ we have that
%     $$||\nabla_\theta \hat\lik^1_J(\theta)(\omega)-\nabla_\theta \hat\lik^1_J(\theta')(\omega)||_\infty < \epsilon.$$


%     Then, by Arzela-Ascoli, there exists a uniformly convergent subsequence. We claim that there is only one subsequential limit. We know that when as the gradient is bounded by Assumption \ref{assump:local-bounded-derivative} we can treat the gradient as a bounded functional. So by Lemma \ref{lem:posterior-proper-weights} the sequence $(\nabla_\theta \hat\lik^1_J(\theta)(\omega))_{J \in \mathbb{N}}$ converges pointwise for $\theta=\phi$ and almost every $\omega \in \Omega$, and there is therefore only one subsequential limit. The sequence therefore converges uniformly to its limit $\lim_{J \to \infty} \nabla_\theta \hat\lik^1_J(\theta)(\omega).$ Therefore, with uniform convergence for the derivatives established, we can swap the limit and derivative, and obtain that for almost every $\omega \in \Omega$, 
%     $$\lim_{J \to \infty} \nabla_\theta \hat\lik^1_J(\theta)(\omega) = \nabla_\theta \lim_{J \to \infty} \hat\lik^1_J(\theta)(\omega).$$

%     From Lemma \ref{lem:lik-proper-weight}, we know that
%     $$\hat{\lik}^1_J(\theta)(\omega) \to \lik(\theta)$$ for almost every $\omega\in\Omega$.

%     We then have that for almost every $\omega \in \Omega$, 
%     $$\lim_{J \to \infty} \nabla_\theta \hat\lik^1_J(\theta)(\omega) = \nabla_\theta \lim_{J \to \infty} \hat\lik^1_J(\theta)(\omega) = \nabla_\theta \lik(\theta),$$
%     as we wanted. The result then follows by the continuous mapping theorem. 
    
% \end{proof}