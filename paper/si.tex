\documentclass[numsec,webpdf,modern,medium,namedate]{oup-authoring-template}
\onecolumn


\usepackage{booktabs}
\newcommand{\description}{}
\usepackage{enumitem}
\usepackage{tabto}

\theoremstyle{thmstyleone}%
\newtheorem{theorem}{Theorem}%  meant for continuous numbers
%%\newtheorem{theorem}{Theorem}[section]% meant for sectionwise numbers
%% optional argument [theorem] produces theorem numbering sequence instead of independent numbers for Proposition
\newtheorem{proposition}[theorem]{Proposition}%
%%\newtheorem{proposition}{Proposition}% to get separate numbers for theorem and proposition etc.
\theoremstyle{thmstyletwo}%
\newtheorem{example}{Example}%
\newtheorem{remark}{Remark}%
\theoremstyle{thmstylethree}%
\newtheorem{definition}{Definition}
\usepackage[doublespacing]{setspace}
\usepackage[fontsize=12pt]{fontsize}


%%%%% NEW MATH DEFINITIONS %%%%%
\usepackage{amsmath,bbm,bm}
\usepackage{amssymb}
\usepackage{amsfonts}
\usepackage{amsthm}
\usepackage{mathtools}

% commands
% global count (no section number)
\newtheorem{thm}{Theorem}%[section]
\newtheorem{lem}{Lemma}
\newtheorem{prop}{Proposition}
\newtheorem{cor}{Corollary}
\newtheorem{conj}{Conjecture}
\newtheorem{aspt}{Assumption}
\newtheorem{claim}{Claim}
\newtheorem{rmk}{Remark}
\newtheorem{commt}{Comment}
\newtheorem{defn}{Definition}

% algorithm
%\usepackage{algorithm, algorithmic}
%\usepackage{algorithm2e}
\usepackage{tabularx}
%\usepackage[table,xcdraw]{xcolor}

% Comments
% \usepackage{xcolor} % already loaded
\newcount\comments  % 0 suppresses notes to selves in text
\comments=1  % TODO: change to 0 for final version
\newcommand{\genComment}[2]{\ifnum\comments=1{\textcolor{#1}{\textsf{\footnotesize #2}}}\fi}
\newcommand{\ed}[1]{\genComment{red}{[EI:#1]}}
\newcommand{\giles}[1]{\genComment{green}{[GH:#1]}}
\newcommand{\kevin}[1]{\genComment{blue}{[KT:#1]}}


% Mark sections of captions for referring to divisions of figures
\newcommand{\figleft}{{\em (Left)}}
\newcommand{\figcenter}{{\em (Center)}}
\newcommand{\figright}{{\em (Right)}}
\newcommand{\figtop}{{\em (Top)}}
\newcommand{\figbottom}{{\em (Bottom)}}
\newcommand{\captiona}{{\em (a)}}
\newcommand{\captionb}{{\em (b)}}
\newcommand{\captionc}{{\em (c)}}
\newcommand{\captiond}{{\em (d)}}


\newcommand\seq[2]{{#1}\!:\!{#2}}
\newcommand\R{\mathbb{R}}
\newcommand\Var{\mathrm{Var}}
\newcommand\var{\Var}
\newcommand\Cov{\mathrm{Cov}}
\newcommand\cov{\Cov}
\newcommand\iid{\mathrm{iid}}
\newcommand\dist{d}
\newcommand\lik{\mathcal{L}}
\newcommand\prob{\mathbb{P}}
\newcommand\E{\mathbb{E}}
\newcommand\loglik{\ell}
\newcommand\process{\texttt{process}}
\newcommand\dimtheta{\mathrm{dim}_{\Theta}}
\newcommand\param{\,;}
\newcommand\giventh\param
\newcommand\given{{\,\vert\,}}
\newcommand\code[1]{\texttt{#1}}
\newcommand\ceil[1]{\lceil #1 \rceil}
\newcommand\floor[1]{\lfloor #1 \rfloor}
\newcommand\1{\bm{1}}


% Highlight a newly defined term
\newcommand{\newterm}[1]{{\bf #1}}


% Figure reference, lower-case.
\def\figref#1{figure~\ref{#1}}
% Figure reference, capital. For start of sentence
\def\Figref#1{Figure~\ref{#1}}
\def\twofigref#1#2{figures \ref{#1} and \ref{#2}}
\def\quadfigref#1#2#3#4{figures \ref{#1}, \ref{#2}, \ref{#3} and \ref{#4}}
% Section reference, lower-case.
\def\secref#1{section~\ref{#1}}
% Section reference, capital.
\def\Secref#1{Section~\ref{#1}}
% Reference to two sections.
\def\twosecrefs#1#2{sections \ref{#1} and \ref{#2}}
% Reference to three sections.
\def\secrefs#1#2#3{sections \ref{#1}, \ref{#2} and \ref{#3}}
% Reference to an equation, lower-case.
\def\eqref#1{equation~\ref{#1}}
% Reference to an equation, upper case
\def\Eqref#1{Equation~\ref{#1}}
% A raw reference to an equation---avoid using if possible
\def\plaineqref#1{\ref{#1}}
% Reference to a chapter, lower-case.
\def\chapref#1{chapter~\ref{#1}}
% Reference to an equation, upper case.
\def\Chapref#1{Chapter~\ref{#1}}
% Reference to a range of chapters
\def\rangechapref#1#2{chapters\ref{#1}--\ref{#2}}
% % Reference to an algorithm, lower-case.
% \def\algref#1{algorithm~\ref{#1}}
% % Reference to an algorithm, upper case.
% \def\Algref#1{Algorithm~\ref{#1}}
% \def\twoalgref#1#2{algorithms \ref{#1} and \ref{#2}}
% \def\Twoalgref#1#2{Algorithms \ref{#1} and \ref{#2}}
% Reference to a part, lower case
\def\partref#1{part~\ref{#1}}
% Reference to a part, upper case
\def\Partref#1{Part~\ref{#1}}
\def\twopartref#1#2{parts \ref{#1} and \ref{#2}}

\def\eps{{\epsilon}}

\def\gN{{\mathcal{N}}}
\def\gX{{\mathcal{X}}}
\def\gY{{\mathcal{Y}}}



% Use the lineno option to display guide line numbers if required.
\usepackage{xr, refcount}
\makeatletter
\newcommand*{\addFileDependency}[1]{% argument=file name and extension
  \typeout{(#1)}
  \@addtofilelist{#1}
  \IfFileExists{#1}{}{\typeout{No file #1.}}
}
\makeatother

\newcommand*{\myexternaldocument}[1]{%
    \externaldocument{#1}%
    \addFileDependency{#1.tex}%
    \addFileDependency{#1.aux}%
}
\externaldocument{ms}





\journaltitle{Journals of the Royal Statistical Society}
\DOI{DOI HERE}
\copyrightyear{XXXX}
\pubyear{XXXX}
\access{Advance Access Publication Date: Day Month Year}
\appnotes{Original article}

\firstpage{1}

%\subtitle{Subject Section}
\title[AD for POMPs]{Web-Based Supporting Materials for Automatic Differentiation Accelerates Inference for Partially Observed Markov Processes}

\author[1,$\ast$]{Kevin Tan}
\author[1]{Giles Hooker}
\author[2]{Edward L. Ionides}
%\author[4]{Fifth Author\ORCID{0000-0000-0000-0000}}
\authormark{Tan et al.}

\address[1]{\orgdiv{Department of Statistics and Data Science}, \orgname{University of Pennsylvania}, \orgaddress{\street{265 South 37th Street, 3rd \& 4th Floors}, \postcode{19104}, \state{Pennsylvania}, \country{United States of America}}}
\address[2]{\orgdiv{Department of Statistics}, \orgname{University of Michigan}, \orgaddress{\street{323 West Hall, 1085 S University Ave}, \postcode{48109}, \state{Michigan}, \country{United States of America}}}

\corresp[$\ast$]{Address for correspondence. Kevin Tan, University of Pennsylvania, Philadelphia, 19104, USA. \href{Email:kevtan@wharton.upenn.edu}{kevtan@wharton.upenn.edu}}

% \received{Date}{0}{Year}
% \revised{Date}{0}{Year}
% \accepted{Date}{0}{Year}
\AtBeginDocument{\setcounter{thm}{\getrefnumber{finalthm}}}
\AtBeginDocument{\setcounter{lem}{\getrefnumber{finallem}}}
\AtBeginDocument{\setcounter{defn}{\getrefnumber{finaldefn}}}
\AtBeginDocument{\setcounter{prop}{\getrefnumber{finalprop}}}
\AtBeginDocument{\setcounter{aspt}{\getrefnumber{finalaspt}}}
\AtBeginDocument{\setcounter{figure}{\getrefnumber{finalfig}}}


\begin{document}

%\addtocounter{thm}{-1}


%\addtocounter{lem}{-1}


%\addtocounter{defn}{-1}


%\addtocounter{prop}{-1}


%\addtocounter{aspt}{-1}



\abstract{Although automatic differentiation (AD) has driven many recent advances in machine learning, partially observed nonlinear stochastic dynamical systems have proved resistant to AD techniques because widely used particle filter algorithms yield an estimated likelihood that is discontinuous in the model parameters. To resolve this, we create a theoretical framework that embeds two existing AD particle filter methods within a new class of algorithms. This new class permits a bias-variance tradeoff and a mean squared error substantially lower than the existing algorithms. This allows us to develop likelihood maximization algorithms suited to the Monte Carlo properties of the AD gradient estimate. Our algorithms require only a differentiable simulator for the latent dynamic system while most previous approaches require access to the system's transition probabilities. Numerical results show that using AD to refine a coarse solution from an iterated filtering algorithm shows substantial improvement on current state-of-the-art methods on a challenging scientific benchmark problem.}


\maketitle



\section{MOP-$\alpha$ Functional Forms}


\label{appendix:functional}


Theorem \ref{thm:mop-functional-forms} follows immediately as a consequence of the following results, Lemmas \ref{lem:mop-1-formula} and \ref{lem:mop-0-formula}.

\begin{lem}
    \label{lem:mop-1-formula}
    Write $\nabla_\theta \hat\ell^\alpha(\theta)$ for the gradient estimate yielded by MOP-$\alpha$ when $\theta=\phi$. Consider the case where we use the after-resampling conditional likelihood estimate so that $\hat\lik(\theta) = \prod_{n=1}^N L_n^{A, \theta, \alpha}$. When $\alpha=1$,
    \begin{equation}
        \nabla_\theta \hat{\ell}^1(\theta) 
        = \frac{1}{J}\sum_{j=1}^J \nabla_\theta \log f_{Y_{1:N}|X_{1:N}}\left(y_{1:N}^* | x_{1:n,j}^{A, F,\theta}\right),
    \end{equation}
    yielding the estimator of \cite{poyiadjis11, scibior21} with the bootstrap filter.
\end{lem}

\begin{proof}
    Consider the case of MOP-$\alpha$ when $\alpha=1$ and $\theta=\phi$. We then have a nice telescoping product property for the after-resampling likelihood estimate:
\begin{equation}
    \hat{\lik}^1(\theta) := \prod_{n=1}^N L_n^{A, \theta, \alpha} = \prod_{n=1}^N L_n^\phi \cdot \frac{\sum_{j=1}^J w_{n,j}^{F,\theta}}{\sum_{j=1}^J w_{n,j}^{P,\theta}}= \prod_{n=1}^N L_n^\phi \cdot \frac{\sum_{j=1}^J w_{n,j}^{F,\theta}}{\sum_{j=1}^J w_{n-1,j}^{F,\theta}} = \left(\frac{1}{J}\sum_{j=1}^J w_{N,j}^{F,\theta}\right) \prod_{n=1}^N L_n^\phi,
\end{equation}
where the third equality follows from the choice of $\alpha=1$, and the fourth equality is the resulting telescoping property. 
The log-derivative identity lets us decompose the score estimate as
\begin{equation}\label{eq:log-derivative-identity}
\nabla_\theta \hat\ell^1(\theta) = \frac{\nabla_\theta \hat\lik^1(\theta)}{\hat\lik^1(\theta)} = \frac{\nabla_\theta\left(\frac{1}{J}\sum_{j=1}^J w_{N,j}^{F,\theta}\right) \prod_{n=1}^N L_n^\phi}{\prod_{n=1}^N L_n^\phi} =  \frac{1}{J}\sum_{j=1}^J \nabla_\theta w_{N,j}^{F,\theta}.
\end{equation}
From (\ref{eq:log-derivative-identity}), we see that the derivative of the log-likelihood estimate is
\begin{equation}\label{eq:eq:log-derivative2}
    \nabla_\theta \hat{\ell}^1(\theta) := \frac{1}{J}\sum_{j=1}^J \nabla_\theta w_{N,j}^{F,\theta}.
\end{equation}
We proceed to decompose (\ref{eq:eq:log-derivative2}).
First, observe that as $\alpha=1$,
\begin{equation}
w_{n,j}^{P,\theta} = w_{n-1,j}^{F,\theta}\frac{g_{n,j}^\theta}{g_{n,j}^\phi} = \prod_{i=1}^n \frac{g_{i,j}^{A,P,\theta}}{g_{i,j}^{A,P,\phi}},
\end{equation}
where we use the $(\cdot)^A$ superscript to denote the ancestral trajectory of the $j$-th prediction or filtering particle at timestep $n$. 
Note that this quantity is the cumulative product of measurement density ratios over the ancestral trajectory of the $j$-th prediction particle at timestep $n$.
We then use the log-derivative identity again, yielding the following expression for the gradient of the log-weights as the sum of the log measurement densities over the ancestral trajectory:
\begin{eqnarray}
 \frac{\nabla_\theta w_{n,j}^{P,\theta}}{w_{n,j}^{P,\theta}} = \nabla_\theta \log w_{n,j}^{P,\theta} &=& \nabla_\theta \log \left(\prod_{i=1}^n \frac{g_{i,j}^{A,P,\theta}}{g_{i,j}^{A,P,\phi}}\right) 
 \\
 &=& \nabla_\theta \sum_{i=1}^n \left(\log g_{i,j}^{A,P,\theta} - \log g_{i,j}^{A,P,\phi}\right)
 \\
 &=& \sum_{i=1}^n \nabla_\theta \log g_{i,j}^{A,P,\theta}.
\end{eqnarray}
This is equal to the gradient of the logarithm of the conditional density of the observed measurements given the ancestral trajectory of the $j$-th prediction particle up to timestep $n$:
\begin{eqnarray}  
\nabla_\theta \sum_{n=1}^N \log g_{n,j}^{A,\theta} &=& \nabla_\theta \log\left(\prod_{n=1}^N g_{n,j}^{A,P,\theta}\right) 
\\
&=&  \nabla_\theta \log\left(\prod_{n=1}^N f_{Y_n|X_n}\left(y_n^* | x_{n,j}^{A, P,\theta}\right)\right)
\\
&=& \nabla_\theta \log f_{Y_{1:N}|X_{1:N}}\left(y_{1:N}^* | x_{1:n,j}^{A, P,\theta}\right).
\end{eqnarray}
Multiplying both sides of the expression by $w_{N,j}^{P,\theta} $ yields an expression for the gradient of the weights at timestep $N$:
\begin{equation}
\nabla_\theta w_{N,j}^{P,\theta} = w_{N,j}^{P,\theta} \sum_{n=1}^N \nabla_\theta \log g_{n,j}^{A,P,\theta} = w_{N,j}^{P,\theta} \nabla_\theta \log f_{Y_{1:N}|X_{1:N}}\left(y_{1:N}^* | x_{1:n,j}^{A, P,\theta}\right).    
\end{equation}
Substituting the above identity into the log-likelihood decomposition obtained earlier in Equation \ref{eq:log-derivative-identity} yields
\begin{equation}
    \nabla_\theta \hat{\ell}^1(\theta) := \frac{1}{J}\sum_{j=1}^J \nabla_\theta w_{N,j}^{F,\theta} =\frac{1}{J}\sum_{j=1}^J \nabla_\theta w_{N,k_j}^{P,\theta} = \frac{1}{J}\sum_{j=1}^J w_{N,k_j}^{P,\theta} \nabla_\theta \log f_{Y_{1:N}|X_{1:N}}\left(y_{1:N}^* | x_{1:n,k_j}^{A, P,\theta}\right).
\end{equation}
Finally, observing that $\theta=\phi$ implies $w_{N,j}^{F,\theta}=1$, we obtain 
\begin{equation}
    \nabla_\theta \hat{\ell}^1(\theta) := \frac{1}{J}\sum_{j=1}^J \nabla_\theta \log f_{Y_{1:N}|X_{1:N}}\left(y_{1:N}^* | x_{1:n,j}^{A, F,\theta}\right).
\end{equation}
This yields the gradient estimators of \cite{poyiadjis11, scibior21} when applied to the bootstrap filter. 
\end{proof}

Note that the variance of the MOP-$\alpha$ log-likelihood estimate scales poorly with $N$ the moment $\theta\neq\phi$. 
This can be seen by observing that
\begin{equation}
    \nabla_\theta \hat{\ell}^1(\theta) := \frac{1}{J}\sum_{j=1}^J \nabla_\theta w_{N,j}^{F,\theta} =\frac{1}{J}\sum_{j=1}^J \nabla_\theta w_{N,k_j}^{P,\theta} = \frac{1}{J}\sum_{j=1}^J w_{N,k_j}^{P,\theta} \nabla_\theta \log f_{Y_{1:N}|X_{1:N}}\left(y_{1:N}^* | x_{1:n,k_j}^{A, P,\theta}\right).
\end{equation}
When $\theta\neq\phi$, we see that $w_{N,k_j}^{P,\theta} = O(c^N)$. 
When $\theta=\phi$, this is a special case of the \cite{poyiadjis11} estimator, which has $O(N^4)$ variance by a property of functionals of the particle filter \cite{delMoral03}. 

\begin{lem}
\label{lem:mop-0-formula}

 Write $\nabla_\theta \hat\ell^\alpha(\theta)$ for the gradient estimate yielded by MOP-$\alpha$ when $\theta=\phi$. Consider the case where we use the after-resampling conditional likelihood estimate so that $\hat\lik(\theta) = \prod_{n=1}^N L_n^{A, \theta, \alpha}$. When $\alpha=0$,
    \begin{equation}
        \nabla_\theta \hat\ell^0(\theta) 
        = \frac{1}{J} \sum_{n=1}^N \sum_{j=1}^J \nabla_\theta \log\left(f_{Y_n|X_{n}}(y_n^*|x_{n,j}^{F, \theta}; \theta)\right),
    \end{equation}
    yielding the estimate of \cite{naesseth18} when applied to the bootstrap filter. 
\end{lem}

\begin{proof}
First, write $$s_{n,j} = \frac{f_{Y_n|X_n}(y_n^*|x_{n,j}^{P, \theta})}{f_{Y_n|X_n}(y_n^*|x_{n,j}^{P, \phi})}$$
as shorthand for the measurement density ratios. 
Observe that, when $\alpha=0,$, the likelihood estimate becomes
\begin{eqnarray}
% \nonumber
    \hat{\lik}^0(\theta) := \prod_{n=1}^N L_n^{A, \theta, \alpha} &=& \prod_{n=1}^N L_n^\phi \cdot \frac{\sum_{j=1}^J w_{n,j}^{F,\theta}}{\sum_{j=1}^J w_{n,j}^{P,\theta}} 
    \\
    &=& \prod_{n=1}^N L_n^\phi \cdot \frac{1}{J}\sum_{j=1}^J s_{n,j} 
    \\
    &=& \prod_{n=1}^N L_n^\phi \cdot \frac{1}{J}\sum_{j=1}^J \frac{f_{Y_n|X_n}(y_n^*|x_{n,j}^{P, \theta})}{f_{Y_n|X_n}(y_n^*|x_{n,j}^{P, \phi})}.
\end{eqnarray}
We lose the nice telescoping property observed in the MOP-$1$ case, but this expression still yields something useful. 
This is because its gradient when $\theta=\phi$ is therefore 
\begin{align}
    \nabla_\theta \hat{\ell}^0(\theta) &:= \sum_{n=1}^N \nabla_\theta \log\left(L_n^\phi \frac{1}{J} \sum_{j=1}^J s_{n,j}\right) \\
    &= \sum_{n=1}^N \frac{\nabla_\theta \left(L_n^\phi \frac{1}{J} \sum_{j=1}^J s_{n,j}\right)}{\left(L_n^\phi \frac{1}{J} \sum_{j=1}^J s_{n,j}\right)} \\
    &= \sum_{n=1}^N \frac{\sum_{j=1}^J \nabla_\theta s_{n,j}}{\sum_{j=1}^J s_{n,j}} \\
    &= \sum_{n=1}^N \frac{1}{J} \sum_{j=1}^J \frac{\nabla_\theta f_{Y_n|X_{n}}(y_n^*|x_{n,j}^{F, \theta}; \theta)}{f_{Y_n|X_{n}}(y_n^*|x_{n,j}^{F, \phi}; \phi)} \\
    &= \frac{1}{J} \sum_{n=1}^N \sum_{j=1}^J \nabla_\theta \log\left(f_{Y_n|X_{n}}(y_n^*|x_{n,j}^{F, \theta}; \theta)\right),
\end{align}
where we use the log-derivative trick in the second equality, observe that $\sum_{j=1}^J s_{n,j} = J$ when $\theta=\phi$ in the fourth equality, and use the log-derivative trick again while noting that $\theta=\phi$ in the fifth equality. This yields the desired result.
\end{proof}






\section{Optimization Convergence Analysis}
\label{appendix:convergence}



The analysis in this section roughly follows the analysis in \cite{mahoney16}, except with the caveat that none of the matrix concentration bounds they use apply here as the particles are dependent. We instead use the concentration inequality from \cite{delMoral11} to bound the gradient and Hessian estimates. In this section, we fix $\omega \in \Omega$ only within each filtering iteration, evaluate Algorithm \ref{alg:mop} at $\theta=\phi$, and analyze Algorithm \ref{alg:ifad} post-iterated filtering.


The convergence analysis in Theorem \ref{thm:mop-convergence} is limited to the case where $-\ell$ is $\gamma$-strongly convex. Though it is true that in a neighborhood of the optimum local asymptotic normality holds and the log-likelihood is strongly convex in this neighborhood, in practice likelihood surfaces for POMPs are often highly nonconvex globally. The convergence to an optimum, local or global, must therefore be sensitive to initialization.  

\subsection{Bounding the Gradient}
%\kevin{Add a few words to explain that the gradient of the log-likelihood w.r.t. parameters is a functional of the particle measure under boundedness assumptions and fixed parameters, appeal to authority and say doucet did it too. }

%\kevin{Gradient linear map, bound conditional likelihoods at time n. Alternatively, convert del moral and doucet expectation bound to high-probability bound. https://arxiv.org/pdf/1905.11546.pdf for trick. Alternatively, concatenate previous states. }

\begin{lem}[Concentration of Measure for Gradient Estimate]
    \label{lemma:grad_bound}
    Consider the gradient estimate obtained by MOP-1, which we know by Theorem \ref{thm:mop-grad-consistency} is consistent for the score, where $\theta = \phi$. For $||\nabla_\theta \hat{\loglik}^1(\theta) - \nabla_\theta \loglik(\theta)||_2$ to be bounded by $\epsilon$ with probability $1-\delta$, we require
    \begin{align}
    J > \max\left\{2G(\theta)\frac{r_N\sqrt{p}}{\epsilon}\left(1+h^{-1}\left(\log\left(\frac{2p}{\delta}\right)\right)\right), 8G(\theta)^2\beta_N^2\frac{p\log(2p/\delta)}{\epsilon^2}\right\},
    \end{align}
    where $NG'(\theta) \leq G(\theta)$ are defined in Assumptions \ref{assump:bounded-measurement} and \ref{assump:local-bounded-derivative}, $h(t) = \frac{1}{2}(t - \log(1+t))$, and $\beta_N$ and $r_N$ are two additional finite model-specific constants that do not depend on $J$, but do depend on $N$ and $p$, as defined in \cite{delMoral11}. 
Equivalently, with probability at least $1-\delta$, it holds that
    \begin{align}
        ||\nabla_\theta \hat\ell^1(\theta) - \nabla_\theta \ell(\theta)||_2 \leq G(\theta)\left(\frac{r_N\sqrt{p}}{J}(1+h^{-1}(\log(2p/\delta))) + \sqrt{\frac{2p\log(2p/\delta)}{J}}\beta_N\right).
    \end{align}
\end{lem}

\paragraph{Remark:} According to \cite{delMoral11}, under some regularity conditions, $r_N$ and $\beta_N$ are linear in the trajectory length $N$. This corresponds to the finding by \cite{poyiadjis11} that the variance of the estimate is at least quadratic in the trajectory length, and their remark that the result of \cite{delMoral03} establishes that the $L_p$ error is bounded by $O(N^2J^{-1/2})$ (equivalently, the variance is bounded by $O(N^4J^{-1})$) after accounting for the sum over timesteps. The MOP-$1$ variance upper bound is therefore in fact $O(N^4)$, in contrast to the MOP-$\alpha$, where $\alpha<1$, upper bound of $O(N)$. 



\begin{proof}


We will seek to use the concentration inequality of \cite{delMoral11} to bound the deviation of the gradient estimate from the gradient of the negative log-likelihood in the sup norm with a union bound. Fix $\theta = \phi$.
From the decomposition in the proof of Lemma \ref{lem:mop-1-formula}, as $w_{N, j}^{F, \theta}=1$ when $\theta=\phi$, we have that
\begin{equation}
\nabla_\theta \hat{\ell}(\theta):=\frac{1}{J} \sum_{j=1}^J \nabla_\theta w_{N, j}^{F, \theta}=\frac{1}{J} \sum_{j=1}^J\sum_{n=1}^N  w_{N, k_j}^{P, \theta} \nabla_\theta \log g_{n,k_j}^{A,\theta} = \frac{1}{J} \sum_{j=1}^J\sum_{n=1}^N \nabla_\theta \log g_{n,k_j}^{A,\theta}.
\end{equation}
Define $\varphi_n^i(x_{n,j}^{F,\theta}) := \frac{\partial}{\partial\theta_i} \log g_{n,k_j}^{A,\theta}$, which is a functional of the filtering particles $x_{n,j}^{F,\theta} = x_{n,k_j}^{P,\theta}$. These are bounded measurable functionals bounded by $G'(\theta)$ by Assumption \ref{assump:bounded-measurement}. Therefore, these have bounded oscillation, satisfying the requirement that $\text{osc} \left(\frac{\partial}{\partial\theta_i} \varphi_i(x_{n,j}^{P,\theta}) \right) \leq G'(\theta)$. Note that \cite{delMoral11} in fact assume $\text{osc}(f) \leq 1$, so we simply scale their bound accordingly.

Now we apply the Hoeffding-type concentration inequality from Del Moral and Rio \cite{delMoral11} and a union bound over each $\varphi_n^i(x_{n,j}^{F,\theta})$, totaling $N$ timesteps and $p$ parameters, to find that
\begin{align}
    \max_{n=1,...,N} \left\lVert\frac{1}{J}\sum_{j=1}^J\nabla_\theta \log g_{n,k_j}^{A,\theta} - \nabla_\theta \ell_n(\theta) \right\rVert_{\infty} \leq G'(\theta)\left(\frac{r_N}{J}(1+h^{-1}(t)) + \sqrt{\frac{2t}{J}}\beta_N \right)
\end{align}
with probability at least $1-2Np\exp(-t)$. Although the above concentration inequality only considers the error from the expectation under the filtering distribution, we invoke the consistency of MOP-$1$ shown in Theorem \ref{thm:mop-grad-consistency} to establish that the expectation under the filtering distribution is in fact the score. It therefore holds that with the same probability, that when summing over $N$, as $NG'(\theta) \leq G(\theta)$, 
\begin{align}
    \left\lVert\frac{1}{J}\sum_{j=1}^J\sum_{n=1}^N\nabla_\theta \log g_{n,k_j}^{A,\theta} - \nabla_\theta \ell(\theta) \right\rVert_{\infty} 
    &\leq G'(\theta)N\left(\frac{r_N}{J}(1+h^{-1}(t)) + \sqrt{\frac{2t}{J}}\beta_N \right)\\
    &\leq G(\theta)\left(\frac{r_N}{J}(1+h^{-1}(t)) + \sqrt{\frac{2t}{J}}\beta_N \right).
\end{align}
We split the $\delta$ failure probability among these $2Np$ terms, to find $\delta\leq2Np\exp(-t)$, and therefore, $t\leq\log(2Np/\delta)$, where $h(t) = \frac{1}{2}(t - \log(1+t))$. 
The two additional model-specific parameters are $\beta_t$ and $r_t$, which do not depend on $J$. 
The analogous bound for the 2-norm follows from scaling the right-hand side by $\sqrt{p}$, to require 
\begin{align}
    ||\nabla_\theta \hat\ell(\theta) - \nabla_\theta \ell(\theta)||_2 \leq G(\theta)\left(\frac{r_N\sqrt{p}}{J}(1+h^{-1}(\log(2p/\delta))) + \sqrt{\frac{2p\log(2p/\delta)}{J}}\beta_N\right).
\end{align}
We therefore need 
\begin{align}
    J > \max\left\{2G(\theta)\frac{r_N\sqrt{p}}{\epsilon}\left(1+h^{-1}\left(\log\left(\frac{2p}{\delta}\right)\right)\right), 8G(\theta)^2\beta_N^2\frac{p\log(2p/\delta)}{\epsilon^2}\right\}.
\end{align}

\end{proof}


\subsection{Bounding Hessian Estimates}

Should one choose to use a second-order method involving a particle Hessian estimate, we provide a guarantee for its positive-definiteness below.

\begin{lem}[Minimum Eigenvalue Bound for Hessian Estimate]
    \label{lemma:hess_bound}
    Assume that the Hessian of the negative log-likelihood $H=\sum_{j=1}^J \E H_j$ has a minimum eigenvalue $0<\gamma<1$, and that $\E \lambda_{\min} (H_j) = \gamma' > 0$. 
    If 
    \begin{equation}
        J > \max\left\{\frac{2r_t(1+h^{-1}(t)) + 2c}{\gamma'}, \frac{2(2t\beta_t^2+c)^2}{\gamma'^2}\right\} \geq  \frac{r_t(1+h^{-1}(t))}{\gamma'} + \sqrt{2tJ}\beta_t/\gamma' + c/\gamma'
    \end{equation}    
    then $\hat{H}(\theta)$ is invertible and positive definite with minimum eigenvalue greater than or equal to $c \in (0, \sum_{j=1}^J \E\lambda_{\min}(H_j))$, with probability at least $1-\exp(-t)$.
\end{lem}
\begin{proof}
Write $\hat{H}(\theta) = \hat{H} = \sum_{j=1}^J H_j$ for the estimate of the negative of the Hessian, where $H_j$ is an element of the outer sum over the $J$ particles.

As the negative log-likelihood is convex, we want to bound the minimum eigenvalue of $\hat{H}(\theta)$ from below with high probability, so that all the eigenvalues of $\hat{H}(\theta)$ are positive with high probability. This ensures that the estimated Hessian is invertible and positive-definite.

It is known that the minimum eigenvalue of a symmetric matrix is concave. Therefore, it suffices to show that the first inequality in the below expression
\begin{equation}
    0 < \sum_{j=1}^J \lambda_{\min} (H_j) \leq  \lambda_{\min}\left(\sum_{j=1}^J H_j\right) = \lambda_{\min} (\hat{H})
\end{equation}
holds with high probability.
We apply the particle Hoeffding concentration inequality from \cite{delMoral11} to find that  
\begin{align}
    \frac{1}{J}\sum_{j=1}^J \lambda_{\min}(H_j) - \E_{\tilde{\pi}_t}\lambda_{\min}(H_j) &= \frac{1}{J}\sum_{j=1}^J \lambda_{\min}(H_j) - \gamma' \geq -\frac{r_t}{J}(1+h^{-1}(t)) - \sqrt{\frac{2t}{J}}\beta_t \\
    \sum_{j=1}^J \lambda_{\min}(H_j) &\geq -r_t(1+h^{-1}(t)) - \sqrt{2tJ}\beta_t + J\gamma',
\end{align}
with probability at least $1-\exp(-t)$. Here, $h(t) = \frac{1}{2}(t - \log(1+t))$. 
The two additional model-specific parameters are $\beta_t$ and $r_t$, which do not depend on $J$. 

We additionally require, for $c \in (0, \sum_{j=1}^J \E\lambda_{\min}(H_j))$,
\begin{align}
    \sum_{j=1}^J \lambda_{\min}(H_j) \geq -r_t(1+h^{-1}(t)) - \sqrt{2tJ}\beta_t + J\gamma' \geq c, \\
    J\gamma' \geq c + r_t(1+h^{-1}(t)) + \sqrt{2tJ}\beta_t.
\end{align}
We therefore require \ed{WE SAY ``REQUIRE'' AT VARIOUS POINTS, BUT THAT SUGGESTS NECESSITY WHEREAS I SUPPOSE WE MEAN SUFFICIENCY. I EDITED ONE OR TWO OF THESE.}
\begin{equation}
J > \max\left\{\frac{2r_t(1+h^{-1}(t)) + 2c}{\gamma'}, \frac{2(2t\beta_t^2+c)^2}{\gamma'^2}\right\} \geq  \frac{r_t(1+h^{-1}(t))}{\gamma'} + \sqrt{2tJ}\beta_t/\gamma' + c/\gamma'    
\end{equation}
for $\hat{H}(\theta)$ to be invertible and positive definite with minimum eigenvalue greater than or equal to $c$ with probability at least $1-\exp(-t)$.

\end{proof}


\subsection{Convergence Analysis of Theorem \ref{thm:mop-convergence}}

\begin{proof}
In this analysis, we largely follow the proof of Theorem 6 in \cite{mahoney16}.
Define $\theta_\eta = \theta_m + \eta p_m$, where $p_m=-(H(\theta_m))^{-1}g(\theta_m)$. 
As in Roosta-Khorasani and Mahoney \cite{mahoney16}, we want to show there is some iteration-independent $\tilde{\eta}>0$ such that the Armijo condition
\begin{equation}
    f(\theta_m+\eta p_m) \leq f(\theta_m) + \eta\beta p_m^Tg(\theta_m),
\end{equation}
holds for any $0< \eta < \tilde{\eta}$ and some $\beta \in (0,1)$.
By an argument found in the beginning of the proof of Theorem 6 in \cite{mahoney16}, we have that choosing $J$ such that $||\nabla_\theta\hat{\ell}(\theta_m) - \nabla_\theta \ell(\theta_m)|| \leq \epsilon$ and $\lambda_{\min}(H(\theta_m)) \geq c>0$ for each $m$, yields
\begin{align}
    f(\theta_\eta)-f(\theta_m) \leq \eta p_m^Tg(\theta_m) + \epsilon\eta||p_m|| + \eta^2 \Gamma ||p_m||^2 / 2,
\end{align}
with probability $1-\delta/2$. 
From now on, we assume that we are on the success event of this high-probability statement. 
Consequently, we have
\begin{equation}
    p_m^Tg(\theta_m) = -p_m^TH(\theta_m)p_m \geq -c||p_m||^2,
\end{equation}
and we can obtain a decrease in the objective. 
Substituting this into the previous expression,
\begin{align}
    f(\theta_\eta)-f(\theta_m) \leq -\eta p_m^TH(\theta_m)p_m + \epsilon\eta||p_m|| + \eta^2 \Gamma ||p_m||^2 / 2,
\end{align}
the Armijo condition becomes
\begin{align}
    -\eta p_m^TH(\theta_m)p_m + \epsilon\eta||p_m|| + \eta^2 \Gamma ||p_m||^2 / 2 &\leq \eta \beta p_m^Tg(\theta_m) = - \eta \beta p_m^TH(\theta_m)p_m \\
    \epsilon||p_m|| + \eta \Gamma ||p_m||^2 / 2 &\leq (1- \beta) p_m^TH(\theta_m)p_m \\
    \epsilon + \eta \Gamma ||p_m|| / 2 &\leq c(1- \beta) ||p_m||.
\end{align}
This holds and guarantees an iteration-independent lower bound if 
\begin{equation}
    \eta \leq \frac{c(1-\beta)}{\Gamma}, \;\; \epsilon \leq \frac{c(1-\beta)}{2\Gamma}||g(\theta_m)|| \leq \frac{c(1-\beta)}{2}||p_m||,
\end{equation}
which is given by our choice of $\eta$.
Now, first note that
\begin{equation}
||g(\theta_m)|| - ||\nabla_\theta f(\theta_m)|| \leq ||g(\theta_m) - \nabla_\theta f(\theta_m)|| \leq \epsilon => ||\nabla_\theta f(\theta_m)|| \geq ||g(\theta_m)|| - \epsilon
\end{equation} 
and
\begin{equation}
||\nabla_\theta f(\theta_m)||-||g(\theta_m)|| \leq ||\nabla_\theta f(\theta_m)-g(\theta_m)|| \leq \epsilon => ||g(\theta_m)|| \geq ||\nabla_\theta f(\theta_m)|| - \epsilon.
\end{equation}
There are now two cases. 
If the algorithm terminates and $||g(\theta_m)|| \leq \sigma \epsilon$, we can derive 
\begin{equation}
    ||\nabla_\theta f(\theta_m)|| \leq ||g(\theta_m)|| + \epsilon = \sigma\epsilon+\epsilon = (\sigma+1)\epsilon.
\end{equation}
If the algorithm does not terminate, then $||g(\theta_m)|| > \sigma \epsilon$. 
Notice that 
\begin{eqnarray}
    \epsilon \geq ||g(\theta_m) - \nabla_\theta f(\theta_m)|| &\geq& ||g(\theta_m)|| - ||\nabla_\theta f(\theta_m)|| 
    \\
    ||\nabla_\theta f(\theta_m)|| + \epsilon &\geq& ||g(\theta_m)|| \geq \sigma \epsilon 
    \\
    ||\nabla_\theta f(\theta_m)|| &\geq& \sigma \epsilon - \epsilon = (\sigma - 1)\epsilon 
    \\
    \frac{||\nabla_\theta f(\theta_m)||}{\sigma-1} &\geq& \epsilon,
\end{eqnarray}
and now 
\begin{eqnarray}
    ||\nabla_\theta f(\theta_m)|| - \epsilon
    &\geq&  ||\nabla_\theta f(\theta_m)|| - \frac{||\nabla_\theta f(\theta_m)||}{\sigma-1} 
    \\
    &=& \left(1-\frac{1}{\sigma-1}\right)||\nabla_\theta f(\theta_m|| 
    \\
    &=& \frac{\sigma-2}{\sigma-1}||\nabla_\theta f(\theta_m)|| 
    \\
    &\geq& \frac{2}{3}||\nabla_\theta f(\theta_m)||.
\end{eqnarray}
Since $||A^{-1}|| = 1/\sigma_{\min}(A)$,
\begin{align}
    p_m^TH(\theta_m)p_m &= (-(H(\theta_m))^{-1}g(\theta_m))^TH(\theta_m)(-(H(\theta_m))^{-1}g(\theta_m)) \\
    &= g(\theta_m)^T(H(\theta_m))^{-1}g(\theta_m) \\
    &\geq \frac{1}{c}||g(\theta_m)||^2 \\
    &\geq \frac{1}{c}(||\nabla_\theta f(\theta_m)|| - \epsilon)^2 \\
    &\geq \frac{4}{9c}||\nabla_\theta f(\theta_m)||^2.
\end{align}
From the assumption that $f$ is $\gamma$-strongly convex, $\gamma I \preceq \nabla_\theta^2 -\ell \preceq \Gamma I$, by an implication of $\gamma$-strong convexity we have
\begin{align}
    f(\theta_m) - f(\theta^*) \leq \frac{1}{2\gamma}||\nabla_\theta f(\theta_m)||^2,
\end{align}
and we put together:
\begin{equation}
    f(\theta_m) - f(\theta^*) \leq \frac{1}{2\gamma}||\nabla_\theta f(\theta_m)||^2 \leq \frac{9c}{4}\frac{1}{2\gamma}p_m^TH(\theta_m)p_m,
\end{equation}
\begin{equation}
    \frac{8\gamma}{9c}(f(\theta_m) - f(\theta^*)) \leq \frac{4}{9c}||\nabla_\theta f(\theta_m)||^2 \leq p_m^TH(\theta_m)p_m,
\end{equation}
\begin{equation}
    f(\theta_m) - f(\theta^*) \leq \frac{9c}{8\gamma}p_m^TH(\theta_m)p_m,
\end{equation}
\begin{equation}
    -\frac{8\gamma}{9c}(f(\theta_m) - f(\theta^*)) \geq -\frac{4}{9c}||\nabla_\theta f(\theta_m)||^2 \geq -p_m^TH(\theta_m)p_m.
\end{equation}
From earlier, as the Armijo condition is fulfilled with our choice of $\eta$ and $\epsilon$,
\begin{align}
    f(\theta_{m+1})-f(\theta_m) &\leq -\eta p_m^TH(\theta_m)p_m + \epsilon\eta||p_m|| + \eta^2 \Gamma ||p_m||^2 / 2 \\
    &\leq -\eta\beta p_m^TH(\theta_m)p_m \\
    &\leq -\eta\beta\frac{8\gamma}{9c}(f(\theta_m) - f(\theta^*)).
\end{align}
Therefore,
\begin{align}
    f(\theta_{m+1}) - f(\theta^*) 
    &= f(\theta_{m+1})-f(\theta_m)+f(\theta_m)- f(\theta^*) \\
    &\leq f(\theta_m)- f(\theta^*) -\eta\beta\frac{8\gamma}{9c}(f(\theta_m) - f(\theta^*)) \\
    &= (1-\eta\beta\frac{8\gamma}{9c})(f(\theta_m) - f(\theta^*)).
\end{align}

\end{proof}


\section{Feynman-Kac Models and Monte Carlo Approximations}
\label{appendix:feynman}


In this section, we introduce the Feynman-Kac convention of \cite{delMoral04} that has since become commonplace \citep{karjalainen23} for the analysis of the particle filter. The mathematical formalization and notation introduced here will be adopted in the remainder of the analysis, in order to prove Theorems \ref{thm:mop-targeting}, \ref{thm:mop-grad-consistency}, and \ref{thm:mop-biasvar}. Let $(\eta_n)_{n=1}^N, (\pi_n)_{n=1}^N, (\rho_n)_{n=1}^N$ be sequences of probability measures on the state space $\gX$. This is the sequence of prediction distributions $f_{X_{n}|Y_{1:n-1}}$, filtering distributions $f_{X_{n}|Y_{1:n}}$, and posterior distributions $f_{X_{1:n}|Y_{1:n}}$ that we seek to approximate with the particle filter. For any measurable bounded functional $h$, we adopt the following functional-analytic notation, borrowed from \cite{delMoral04, chopin20, karjalainen23}. We choose our specific choice of notation and definitions to be in line with that of \cite{karjalainen23}. 


\paragraph{\bf Markov kernels and the process model:} A Markov kernel $M$ with source $\gX_1$ and target $\gX_2$ is a map $M: \gX_1 \times \mathcal{B}(\gX_2) \to [0,1]$ such that for every set $A \in \mathcal{B}(\gX_2)$ and every point $x \in \gX_1$, the map $x \mapsto M(x, A)$ is a measurable function of $x$, and the map $A \mapsto M(x,A)$ is a probability measure on $\gX_2$. The quantity $M(x,A)$ can be thought of as the probability of transitioning to the set $A$ given that we are at the point $x$. If this yields a density, this then corresponds to the process density $f_{X_{2}|X_1}$ conditional on $x$ and integrated over $A$.  

\paragraph{\bf Markov kernels and measures:} For any measure $\eta$, any Markov kernel $M$ on $\gX$, any point $x \in \gX$ and any measurable subset $A \subseteq \gX$, let 
\begin{align}
    \eta(h) &= \int h \, d\eta = \int h(x) \eta(dx), \\(\eta M)(A) &= \int \eta(dx)M(x,A), \\
    (Mh)(x) &= \int M(x, dy) h(y).
\end{align}

\paragraph{\bf Compositions of Markov kernels:} The composition of a Markov kernel $M_1$ with another Markov kernel $M_2$ is another Markov kernel, given by 
\begin{equation}
 (M_1M_2)(x, A) = \int M_1(x, dy) M_2(y, A).
\end{equation}

\paragraph{\bf Total variation distance:} The total variation distance between two measures $\mu$ and $\nu$ on $\gX$ is
\begin{equation}
\|\mu-\nu\|_{\mathrm{TV}}=\sup _{\|h\|_{\infty} \leq 1 / 2}|\mu(\phi)-\nu(\phi)|=\sup _{\operatorname{osc}(h) \leq 1}|\mu(h)-\nu(h)|.    
\end{equation}

\paragraph{\bf Dobrushin contraction:} The Dobrushin contraction coefficient $\beta_{\text{TV}}$ of a Markov kernel $M$ is given by
\begin{equation}
\beta_{\mathrm{TV}}(M)=\sup _{x, y \in \gX}\big\|M(x, \cdot)-M(y, \cdot)\big\|_{\mathrm{TV}}=\sup _{\mu, \nu \in \mathcal{P}, \mu \neq \nu} \frac{\|\mu M-\nu M\|_{\mathrm{TV}}}{\|\mu-\nu\|_{\mathrm{TV}}}.    
\end{equation}

\paragraph{\bf Potential functions and the measurement model:} A potential function $G : \gX \to [0,\infty)$ is a non-negative function of an element of the state space $x \in \gX$. 
In our case, this corresponds to the measurement model, and in our previous notation is written as $g_{n,j} = f_{Y_n|X_n}(y_n^*|x_{n,j}^F) = G_n(x_{n,j}^F)$, where in a slight abuse of notation we suppress the dependence on $\theta$ for notational simplicity. 
Note that $G_n(\cdot) = f_{Y_n|X_n}(y_n^*|\;\cdot\;)$ is the conditional density of the observed measurement at time $n$, where we condition on the filtering particle $x_{n,j}^F$ as an element of the state space. 

\paragraph{\bf Feynman-Kac models:} A Feynman-Kac model on $\gX$ is a tuple $(\pi_0, (M_n)_{n=1}^N, (G_n)_{n=1}^N)$ of an initial probability measure on the state space $\pi_0$, a sequence of transition kernels $(M_n)_{n=1}^N$, and a sequence of potential functions $(G_n)_{n=1}^N$. In the notation used in the main text, this corresponds to the starting distribution $f_{X_0}$, the sequence of transition densities $f_{X_{n}|X_{n-1}}$, and the measurement densities $f_{Y_n|X_n}$. This induces a set of mappings from the set of probability measures on $\gX$ to itself, $\mathcal{P}(\gX) \to \mathcal{P}(\gX)$, as follows:
\begin{itemize}
    \item The update from the prediction to the filtering distributions is given by 
    \begin{equation}
    \pi_n(dx) = \Psi_n(\eta_n)(dx) = \frac{G_n(x)\cdot\eta_n(dx)}{\eta_n(G_n)}.
    \end{equation}
    \item The map from the prediction distribution at timestep $n$ to timestep $n+1$ is given by 
    \begin{equation}
 \Phi_{n+1}(\eta_n) = \Psi_n(\eta_n) M_{n+1}.       
    \end{equation}
    \item The composition of maps between prediction distributions yields the map from the prediction distribution at time $k$ to the prediction distribution at time $n$ where $k \leq n$,
    \begin{equation}
 \Phi_{k,n} = \Phi_n \circ ... \circ \Phi_{k+1}.       
    \end{equation}
\end{itemize}
\paragraph{\bf The particle filter:} The particle filter then yields a Monte Carlo approximation to the above Feynman-Kac model, via a sequence of mixture Dirac measures. When one resamples at every timestep, the prediction measure at timestep $n$ is then given by 
\begin{equation}
\eta_n^J = \frac{1}{J}\sum_{j=1}^J \delta_{x_{n,j}^P},
\end{equation}
and the filtering measure at timestep $n$ is given by
\begin{equation}
\pi_n^J = \frac{\sum_{j=1}^J g_{n,j} \delta_{x_{n,j}^P}}{\sum_{j=1}^J g_{n,j}} \approx \frac{1}{J} \sum_{j=1}^J \delta_{x_{n,j}^F}.
\end{equation}
In a slight abuse of notation, we will identify $x_{n, 1:J}^P \equiv \eta_n^J$, and $x_{n, 1:J}^F \equiv \pi_n^J$. 
As in \cite{karjalainen23}, one can view this as an inhomogenous Markov process evolving on $\gX^{J}$. The corresponding Markov transition kernel is then 
\begin{equation}
\textbf{M}_n(x_{n-1, 1:J}^P, \cdot) = \left(\Phi_{n}\left(\eta_n^J\right)\right)^{\otimes J} = \left(\Phi_{n}\left(\frac{1}{J}\sum_{j=1}^J \delta_{x_{n,j}^P}\right)\right)^{\otimes J},
\end{equation}
and the composition of Markov kernels on particles from timestep $n$ to timestep $n+k$ is written 
\begin{equation}
\textbf{M}_{n, n+k} = \textbf{M}_{n+k}\circ ...\circ \textbf{M}_n.
\end{equation}
One may wonder why \cite{karjalainen23} require this process to evolve on $\gX^{J}$. This is because at every timestep $n$, we in fact draw $X_{n, j}^P | \{X_{n-1, 1:J}^P = x_{n-1, 1:J}^P\} \sim \textbf{M}_n(x_{n-1, 1:J}^P, \cdot) = \eta_0^{\otimes J} \textbf{M}_{0,n}$ for $j=1,...,J$. 

\paragraph{\bf Forgetting of the particle filter:} 
The above formalization yields a result from \cite{karjalainen23} on the forgetting of the particle filter that we require for our analysis of the bias, variance, and error of MOP-$\alpha$.  
That is, \cite{karjalainen23} show that
\begin{equation}
\beta_{\mathrm{TV}}\left(\mathbf{M}_{n, n+k}\right) \leq(1-\epsilon)^{\lfloor k /(O(\log J))\rfloor},
\end{equation}
for some $\epsilon$ dependent on $\bar{G}, \underbar{G}, \bar{M}, \underbar{M}$ in Assumptions \ref{assump:bounded-measurement} and \ref{assump:bounded-process}. As a result, the mixing time of the particle filter is only on the order of $O(\log(J))$ timesteps. 


\vspace{3mm}

Equipped with the above formalisms and results, we are now in a position to provide guarantees on the performance of MOP-$\alpha$ itself. 

\section{A Strong Law of Large Numbers for Triangular Arrays of Particles With Off-Parameter Resampling}
\label{appendix:targeting}


\begin{defn}[Targeting]
    A random vector $(X, w)$ drawn from a distribution $g$ \textbf{targets} the distribution $\pi$ if for any measurable and bounded functional $h$,
\begin{equation}
    E_g\{h(X) \cdot w\}=E_\pi\{h(X)\}
\end{equation}  
    A set of particles $(X^J_j, w^J_j), j=1,2, \ldots,J$, \textbf{targets} $\pi$ if for any measurable and bounded functional $h$,
\begin{equation}
    \frac{\sum_{j=1}^J h(X^J_j) w^J_j}{\sum_{j=1}^J w^J_j} \stackrel{a.s.}{\to} E_\pi(h(X))
\end{equation}
as $J \to \infty$.
\end{defn}
\cite{chopin04} \ed{and \cite{liu01}????} asserted without proof that common particle filter algorithms target the filtering distribution, $f_{X_{1:N}|Y_{1:N}}$, in this sense.
\ed{\cite{chopin20} proved a related result assuming bounded densities.
  We follow a similar approach to \cite{chopin20}, based on a strong laws of large numbers for triangular arrays.
  Triangular array strong laws do not hold without an additional regularity condition such as boundedness.}
\ed{I PROPOSE ADDING SOMETHING LIKE THIS, PARTLY TO MOTIVATE THE RESULTS, PARTLY TO BE KINDER TO CHOPIN.}  
In order to prove the consistency of our variation on the particle filter, we now present three helper lemmas.
The first follows from standard importance sampling arguments, the second from integrating out the marginal, and the third from Bayes' theorem. 
We state Lemma~\ref{lem:change-measure-proper-weights} assuming multinomial resampling, which is convenient for the proof though other resampling strategies may be preferable in practice.

\begin{lem}[Change of Weight Measure]
    \label{lem:change-measure-proper-weights}
    Suppose that $\{(\tilde X_j^J,U_j^J),j=1,\dots,J\}$ targets $f_X$. Now, let $\{(Y_j^J,V_j^J),j=1,\dots,J\}$ be a multinomial sample drawn from $\{(\tilde X_j^J,U_j^J)\}$ where $(\tilde X_j^J,U_j^J)$ is represented, on average, proportional to $\pi^J_j J$ times. Write
    \[
    (Y_j^J,V_j^J) = \big(\tilde X^J_{a(j)},U^J_{a(j)}/\pi^J_{a(j)}\big),
    \]
    where $a$ is called the ancestor function. If the importance sampling weights $U_j/\pi_j$ are bounded, then $\{(Y^J_j,V^J_j),j=1,\dots,J\}$ targets $f_X$.
\end{lem}

\begin{proof}


    Note that as the $Y_j^J$ are a subsample from $X_j^J$, $h$ can be a function of $Y$ as well as it is one for $X$. We then expand
    $$\frac{\sum_j h(Y_j^J) V_j^J}{\sum_j V_j^J} = \frac{\sum_j h(\tilde X_{k_j}^J)\frac{U_{k_j}^J}{\pi_{k_j}^J}}{\sum_j \frac{U_{k_j}^J}{\pi_{k_j}^J}}.$$
    By hypothesis,
    $$\frac{\sum_j h(\tilde X_j^J)U_j^J}{\sum_j U_j^J} \stackrel{a.s.}{\to} \E_{f_X}[f(X)].$$

    We want to show
    \begin{equation}\label{eq:lemma1:h}
    \frac{\sum_j h(X_{k_j}^J)\frac{U_{k_j}^J}{\pi_{k_j}^J}}{\sum_j \frac{U_{k_j}^J}{\pi_{k_j}^J}} - \frac{\sum_j h(\tilde X_j^J)U_j^J}{\sum_j U_j^J} \stackrel{a.s.}{\to} 0.
    \end{equation}
    For this, it is sufficient to show that
   $$ \sum_j h(\tilde X_{k_j}^J)\frac{U_{k_j}^J}{\pi_{k_j}^J}
    -  \sum_j h(\tilde X_j^J)\frac{U_j^J}{\pi_j^J}\pi_j^J \stackrel{a.s.}{\to} 0 $$
    since an application of this result with $h(x)=1$ provides almost sure convergence of the denominator in (\ref{eq:lemma1:h}).
    Write $g(\tilde X_j^J) = h(\tilde X_j^J)\frac{U_{k_j}^J}{\pi_{k_j}^J}$. We therefore need to show that 
    $$\sum_j Z_j^J := \sum_j \left(g(\tilde X_{k_j}^J) -  g(\tilde X_j^J) \pi_j^J \right) \stackrel{a.s.}{\to} 0.$$

    Because the functional $h$ and importance sampling weights $u_{k_j}^J/\pi_{k_j}^J$ are bounded, we have that $\E[\left(Z_j^J\right)^4] < \infty$. We can then follow the argument of \cite{chopin20} from this point on, where 
    {THE Zs ARE ONLY CONDITIONALLY INDEPENDENT, SO WE COULD PUT IN THE CONDITIONING EVENT (I.E., THE Xs AND Us) HERE TO BE MORE ACCURATE}
    $$\E\left[\left(\sum_j Z_j^J\right)^4\right] 
    = J\E[(Z_1^J)^4] + 3J(J-1)\left(\E[(Z_j^J)^2]\right) \leq CJ^2,$$ for some $C>0$. By Markov, 
    $$\mathbb{P}\left(\left|\frac{1}{J} \sum_{j=1}^J Z_j^J\right|>\epsilon\right) 
    \leq \frac{1}{\epsilon^4J^4 } 
    \mathbb{E}\left[\left(\sum_{j=1}^J Z_j^J\right)^4\right] \leq \frac{C}{\epsilon^4J^2},$$
    and as these terms are summable we can apply Borel-Cantelli to conclude that these deviations happen only finitely often for every $\epsilon>0,$ giving us the almost-sure convergence for
    
    $$ \sum_j h(X_{k_j}^J)\frac{u_{k_j}^J}{\pi_{k_j}^J}
    -  \sum_j h(X_j^J)\frac{u_j^J}{\pi_j^J}\pi_j^J \stackrel{a.s.}{\to} 0.$$ 

    Similarly, we also have that
    $$ \sum_j \frac{u_j^J}{\pi_j^J}\pi_j^J
    - \sum_j \frac{u_{k_j}^J}{\pi_{k_j}^J} \stackrel{a.s.}{\to} 0,$$

    and the result is proved. 


    \kevin{MAY HAVE TO DEAL WITH CONDITIONING ON THE FILTRATION AT THE LAST TIMESTEP, AND THEN USING THE TOWER PROPERTY. EXPECTATION WILL BE THE SAME REGARDLESS FOR THE FOURTH MOMENT.}

    
    
   % Let $h$ be an integrable function. 
    %\begin{align*}
    %    \E\left[\sum_{j=1}^J h(Y_j) v_j\right] 
    %    &= \sum_{j=1}^J \E\left[h(X_{a(j)}) \frac{u_{a(j)}}{\pi_{a(j)}}\right] \\
    %    &= \sum_{j=1}^J \E\left[h(X_{j}) u_{j}\right],
    %\end{align*}
    %and similarly for the denominator. Now, the numerator and denominator of the reweighted quantity have the same expectation as the numerator and denominator of the original quantity, so they must converge to $ E_\pi(h(X))$ almost surely as well. 
\end{proof}


\textbf{Remark:} Note that Lemma \ref{lem:change-measure-proper-weights} permits $\pi_{1:J}$ to depend on $\{(X_j,u_j)\}$ as long as the resampling is carried out independently of $\{(X_j,u_j)\}$, conditional on $\pi_{1:J}$.

\begin{lem}[Particle Marginals]
    \label{lem:marginal-proper-weights}
    Suppose that $\{(\tilde X_j^J,U_j^J),j=1,\dots,J\}$ targets $f_X$. Also suppose that $\tilde Z_j^J \sim f_{Z|X}(\cdot | \tilde X_j^J)$ where $f_{Z|X}$ is a conditional probability density function corresponding to a joint density $f_{X,Z}$ with marginal densities $f_X$ and $f_Z$. Then, if the $U_j^J$ are bounded, $\{(\tilde Z_j^J,U_j^J)\}$ targets $f_Z$.
\end{lem}
\begin{proof}
    We want to show that, for any measurable bounded $h$, 
     $$\frac{\sum_j h(\tilde Z_j^J) \, U_j^J}{\sum_j U_j^J} \stackrel{a.s.}{\to} \E_{f_Z}[h(Z)] = \E_{f_X}\big[\E_{f_{Z|X}}[h(Z) | X]\big].$$
    By assumption, for any measurable and bounded functional $g$ with domain $\gX$,
    \begin{equation}\label{eq:lemma2:g}
    \frac{\sum_j g(\tilde X_j^J)\, U_j^J}{\sum_j U_j^J} \stackrel{a.s.}{\to} \E_{f_X}[g(X)].
    \end{equation}
    Let $\bar{U}_j^J = \frac{J U_j^J}{\sum_j U_j^J}$. Examine the numerator and denominator of the quantity $$\frac{J^{-1}\sum_j h(\tilde Z_j^J) \, U_j^J}{J^{-1}\sum_j U_j^J} = \frac{J^{-1}\sum_j h(\tilde Z_j^J) \, \bar{U}_j^J}{J^{-1}\sum_j \bar{U}_j^J}.$$
    The denominator converges to $1$ almost surely. The numerator, on the other hand, is
    $$\frac{1}{J}\sum_j h(\tilde Z_j^J)\, \bar{U}_j^J,$$
    by the same fourth moment argument to the above lemma converges almost surely to the limit of its expectation
    \begin{eqnarray*}        
    \lim_{J\to\infty} \E\left[\frac{1}{J}\sum_j h(\tilde Z_j^J)\, \bar{U}_j^J\right] 
    &=& \lim_{J\to\infty} \E \left[\frac{1}{J}\sum_j  
      \E\left[ h(\tilde Z_j^J)\, \bar{U}_j^J\big|\tilde X_j^J, \bar U_j^J\right]
    \right]
    \\
    &=& \lim_{J\to\infty} \E \left[\frac{1}{J}\sum_j  \E\big[h(Z) \big| X=\tilde X_j^J \big] \, \bar U_j^J\right].
    \end{eqnarray*}
    Applying (\ref{eq:lemma2:g}) with $g(x) = \E\big[h(Z) | X=x\big]$, the average on the right hand side converges almost surely to $\E\big\{\E[h(Z)|X]\big\}=\E[h(Z)].$
    It remains to swap the limit and expectations. We can do so with the bounded convergence theorem, and therefore obtain   
    $$\frac{1}{J}\sum_j h(\tilde Z_j^J)\,\bar{U}_j^J \stackrel{a.s.}{\to} \E_{f_Z}[h(Z)].$$ 
    
\end{proof}

\begin{lem}[Particle Posteriors]
    \label{lem:posterior-proper-weights}
    Suppose that $\{(X_j^J,U_j^J),j=1,\dots,J\}$ targets $f_X$. Also suppose that $(X^{\prime J}_j,U^{\prime J}_j) = \big(X_j^J,U_j^J \cdot f_{Z|X}(z^*|X_j^J)\big)$. Then, if $U_j^J \cdot f_{Z|X}(z^*|X_j^J)\big)$ and $U_j^J \cdot f_{Z|X}(z^*|X_j^J)\big) / f_Z(z^*)$ are bounded, $\{(X^{\prime J}_j,U^{\prime J}_j)\}$ targets $f_{X|Z}(\cdot | z^*)$.
\end{lem}

\begin{proof}

    Again, we want to show that
     $$\frac{\sum_j h(X_j^J) \cdot U_j^J \cdot f_{Z|X}(z^*|X_j^J)}{\sum_j U_j^J \cdot f_{Z|X}(z^*|X_j^J)} \stackrel{a.s.}{\to} \E_{f_{X|Z}}[h(X)|z^*].$$

    We already have that for any measurable bounded $g$,
    \begin{equation}
        \frac{\sum_j g(X_j^J)U_j^J}{\sum_j U_j^J} \stackrel{a.s.}{\to} \E_{f_X}[g(X)].
        \label{eq:lemma-posterior-hypothesis}
    \end{equation}
    
    Consider the following:
    \begin{equation}
    \frac{J^{-1} \sum_j h(X_j^J) f_{Z|X}(z^*|X_j^J) {U}_j^{J}}{J^{-1} \sum_j {U}_j^{J}}
    \times \left( \frac{J^{-1} \sum_j f_{Z|X}(z^*|X_j^J) {U}_j^{J}}{J^{-1} \sum_j {U}_j^{J}} \right)^{-1}.
    \end{equation}

    We will apply Equation \ref{eq:lemma-posterior-hypothesis} to the numerator and the denominator in the ratio above individually. The numerator converges to 
    $$\frac{J^{-1} \sum_j h(X_j^J) f_{Z|X}(z^*|X_j^J) {U}_j^{J}}{J^{-1} \sum_j {u}_j^{J}} \stackrel{a.s.}{\to} \E_{f_X}[h(X)f_{Z|X}(z^*|X)],$$
    while the reciprocal of the denominator converges to 
    $$ \frac{J^{-1} \sum_j f_{Z|X}(z^*|X_j^J) {U}_j^{J}}{J^{-1} \sum_j {U}_j^{J}}  \stackrel{a.s.}{\to} \E_{f_X}[f_{Z|X}(z^*|X)] = f_Z(z^*).$$

    
    But it holds that
    $$\frac{\E_{f_X}[h(X)f_{Z|X}(z^*|X)]}{f_Z(z^*)} = \E_{f_X}\left[\frac{h(X)f_{Z|X}(z^*|X)}{f_Z(z^*)}\right] = \E_{f_X}\left[h(X)\frac{f_{X|Z}(X|z^*)}{f_X(X)}\right] =  \E_{f_{X|Z}}[h(X)|z^*],$$
    and we have the desired result.
    
    
    % Write $\bar{U}_j^{J} = \frac{J U_j^J}{\sum_j U_j^J}$. Define the weights $U^{' J}_j := U_j^J \cdot f_{Z|X}(z^*|X_j^J),$ and the corresponding self-normalized weights $\bar{U}_j^{' J} := \frac{J U^{' J}_j}{\sum_j U^{' J}_j}.$ The denominator of the below quantity 
    % $$ \frac{J^{-1} \sum_j h(X_j^J) \bar{U}_j^{' J}}{J^{-1} \sum_j \bar{U}_j^{' J}}$$
    % converges to $1$, while the numerator, by the same fourth moment argument as the first lemma, converges almost surely to the limit of its expectation, which is
    % \begin{align*}
    %     \lim_{J\to\infty} \E\left[J^{-1} \sum_j h(X_j^J) \bar{u}_j^{' J} \right]
    %     &= \lim_{J\to\infty} \E\left[J^{-1} \sum_j h(X_j^J) \frac{J u_j^J \cdot f_{Z|X}(z^*|X_j^J)}{\sum_j u_j^J \cdot f_{Z|X}(z^*|X_j^J)}\right] \\
    %     &= \lim_{J\to\infty} \E\left[J^{-1} \sum_j h(X_j^J) \frac{J \frac{u_j^J \cdot f_{Z|X}(z^*|X_j^J)}{f_Z(z^*)}}{\sum_j \frac{u_j^J \cdot f_{Z|X}(z^*|X_j^J)}{f_Z(z^*)}}\right]\\
    %     &= \lim_{J\to\infty} \E\left[ \sum_j h(X_j^J) \frac{\frac{u_j^J \cdot f_{Z|X}(z^*|X_j^J)}{f_Z(z^*)}}{\sum_j \frac{u_j^J \cdot f_{Z|X}(z^*|X_j^J)}{f_Z(z^*)}}\right]\\
    %     &= \E_{X|Z}[h(X)|z^*],
    % \end{align*}

    % where we can swap the limit and expectations with the bounded convergence theorem.
\end{proof}


\begin{lem}[Change of Particle Marginal Measure]
    \label{lem:change-marginal-proper-weights}
    Suppose that $\{(\tilde X_j^J,U_j^J),j=1,\dots,J\}$ targets $f_X$. Also suppose that $\tilde Z_j^J \sim f_{Z|X}(\cdot | \tilde X_j^J)$ where $f_{Z|X}$ is a conditional probability density function corresponding to a joint density $f_{X,Z}$ with marginal densities $f_X$ and $f_Z$. Further, let $f_{Y|X}$ be a conditional probability density function corresponding to a joint density $f_{X,Y}$ with marginal densities $f_X$ and $f_Y$. Then, if the $U_j^J$ and $V_j^J := U_j^Jf_{Y|X}(\tilde{Z}_j^J|\tilde{X}_{j}^J)/f_{Z|X}(\tilde{Z}_j^J|\tilde{X}_{j}^J)$ are bounded, $\{(\tilde Z_j^J,U_j^J)\}$ targets $f_Z$, and $\{(\tilde Z_j^J,V_j^J)\}$ targets $f_Y$.
\end{lem}
\begin{proof}
    From Lemma \ref{lem:marginal-proper-weights}, for any measurable bounded $g$, 
     $$\frac{\sum_j g(\tilde Z_j^J) \, U_j^J}{\sum_j U_j^J} \stackrel{a.s.}{\to} \E_{f_Z}[g(Z)] = \E_{f_X}\big[\E_{f_{Z|X}}[g(Z) | X]\big],$$
     and by assumption, for any measurable bounded $g$, 
    $$\frac{\sum_j g(\tilde X_j^J)U_j^J}{\sum_j U_j^J} \stackrel{a.s.}{\to} \E_{f_X}[g(X)].$$

    We in fact want to show that 
    $$\frac{\sum_j h(\tilde Z_j^J) \, V_j^J}{\sum_j V_j^J} \stackrel{a.s.}{\to} \E_{f_Y}[h(Y)] = \E_{f_X}\left[\E_{f_{Z|X}}\left[h(Z)\frac{f_{Y|X}(Z|X)}{f_{Z|X}(Z|X)} \Bigg| X\right]\right].$$

    Consider the following:
    \begin{equation}
    \frac{J^{-1} \sum_j h(Z_j^J) \frac{f_{Y|X}(\tilde Z_j^J|\tilde X_j^J)}{f_{Z|X}(\tilde Z_j^J|\tilde X_j^J)} {U}_j^{J}}{J^{-1} \sum_j {U}_j^{J}}
    \times \left( \frac{J^{-1} \sum_j \frac{f_{Y|X}(\tilde Z_j^J|\tilde X_j^J)}{f_{Z|X}(\tilde Z_j^J|\tilde X_j^J)} {U}_j^{J}}{J^{-1} \sum_j {U}_j^{J}} \right)^{-1}.
    \end{equation}

    We will apply Lemma 2 to the numerator and the denominator in the ratio above individually. The numerator converges to 
    $$\frac{J^{-1} \sum_j h(Z_j^J)  \frac{f_{Y|X}(\tilde Z_j^J|\tilde X_j^J)}{f_{Z|X}(\tilde Z_j^J|\tilde X_j^J)} {U}_j^{J}}{J^{-1} \sum_j {U}_j^{J}} \stackrel{a.s.}{\to} \E_{f_Z}\left[h(Z) \frac{f_{Y|X}(Z|X)}{f_{Z|X}(Z|X)}\right] = \E_{f_X}\left[\E_{f_{Z|X}}\left[h(Z)\frac{f_{Y|X}(Z|X)}{f_{Z|X}(Z|X)}\Bigg|X\right]\right],$$
    while the reciprocal of the denominator converges to 
    $$ \frac{J^{-1} \sum_j \frac{f_{Y|X}(\tilde Z_j^J|\tilde X_j^J)}{f_{Z|X}(\tilde Z_j^J|\tilde X_j^J)}{U}_j^{J}}{J^{-1} \sum_j {U}_j^{J}}  \stackrel{a.s.}{\to} \E_{f_Z}\left[ \frac{f_{Y|X}(Z|X)}{f_{Z|X}(Z|X)}\right] = \E_{f_X}\left[\E_{f_{Z|X}}\left[\frac{f_{Y|X}(Z|X)}{f_{Z|X}(Z|X)}\Bigg|X\right]\right] .$$

    The numerator is exactly what we desire, so it remains to show that the reciprocal of the denominator converges to $1$. Applying Fubini, we have that
    \begin{align*}
        \E_{f_X}\left[\E_{f_{Z|X}}\left[\frac{f_{Y|X}(Z|X)}{f_{Z|X}(Z|X)}\Bigg|X\right]\right] &= \int_x\left(\int_z \frac{f_{Y|X}(z|x)}{f_{Z|X}(z|x)}f_{Z|X}(z|x)dz \right)f_X(x) dx \\
        &= \int_x\int_z f_{Y|X}(z|x) f_X(x) dzdx \\
        &= \int_z\int_x f_{Y|X}(z|x) f_X(x) dxdz \\
        &= \int_z f_{Y}(z) dz \\
        &= 1,
    \end{align*}
    and we are done. 

    
\end{proof}


\begin{prop}[MOP-1 Targets the Posterior]
    When $\alpha=1$ or $\phi=\theta$, MOP-$\alpha$ targets the posterior. 
\end{prop}
\begin{proof}
    When $\theta=\phi$, regardless of the value of $\alpha$, the ratio $\frac{g_{n,j}^\theta}{g_{n,j}^\phi}=1,$ and this reduces to the vanilla particle filter estimate.

    When $\alpha=1$, and $\theta\neq\phi,$ the proof is as follows. Recursively applying Lemmas \ref{lem:change-measure-proper-weights}, \ref{lem:marginal-proper-weights}, and \ref{lem:posterior-proper-weights}, we obtain that 
    %to step~\ref{mop:step1}, Lemma~2 step~ {mop:weight:update} and Lemma~3 to step~\ref{mop:step2} we obtain that
    the MOP-1 filter targets the posterior.
    Specifically, suppose inductively that $\big\{\big(X^{F,\theta}_{n-1,j},w^{F,\theta}_{n-1,j}\big)\big\}$ is properly weighted for $f_{X_{n-1}|Y_{1:n-1}}(x_{n-1}|y^*_{1:n-1};\theta)$.
    Then, Lemma \ref{lem:marginal-proper-weights} tells us that $\big\{\big(X^{P,\theta}_{n,j},w^{P,\theta}_{n,j}\big)\big\}$ targets $f_{X_{n}|Y_{1:n-1}}(x_{n}|y^*_{1:n-1};\theta)$.
    Lemma \ref{lem:posterior-proper-weights} tells us that $\big\{\big(X^{P,\theta}_{n,j},w^{P,\theta}_{n,j} g^\theta_{n,j} \big)\big\}$ therefore targets  $f_{X_{n}|Y_{1:n}}(x_{n}|y^*_{1:n};\theta)$.
    Lemma \ref{lem:change-measure-proper-weights} guarantees that the resampling rule, given by 
    \[
    \big(X^{F,\theta}_{n,j},w^{F,\theta}_{n,j}\big) = \big(X^{P,\theta}_{n,a(j)}, w^{P,\theta}_{n,a(j)} g^\theta_{n,a(j)}\big/ g^\phi_{n,a(j)}\big),
    \]
    with resampling weights proportional to $g^\phi_{n,j}$, therefore also targets $f_{X_{n}|Y_{1:n}}(x_{n}|y^*_{1:n};\theta)$.
\end{proof}

\begin{prop}[DOP-1 Targets the Posterior]
    When $\alpha=1$ or $\phi=\theta$, DOP-$\alpha$ targets the posterior. 
\end{prop}
\begin{proof}
    When $\theta=\phi$, regardless of the value of $\alpha$, the ratio $\frac{g_{n,j}^\theta h_{n,j}^\theta}{g_{n,j}^\phi h_{n,j}^\phi}=1,$ and this reduces to the vanilla particle filter estimate.

    When $\alpha=1$, and $\theta\neq\phi,$ the proof is as follows. Recursively applying Lemmas \ref{lem:change-measure-proper-weights}, \ref{lem:posterior-proper-weights}, and \ref{lem:change-marginal-proper-weights}, we obtain that 
    %to step~\ref{mop:step1}, Lemma~2 step~ {mop:weight:update} and Lemma~3 to step~\ref{mop:step2} we obtain that
    the DOP-1 filter targets the posterior.
    Specifically, suppose inductively that $\big\{\big(X^{F,\phi}_{n-1,j},w^{F,\theta}_{n-1,j}\big)\big\}$ is properly weighted for $f_{X_{n-1}|Y_{1:n-1}}(x_{n-1}|y^*_{1:n-1};\theta)$.
    Then, Lemma \ref{lem:change-marginal-proper-weights} tells us that $\big\{\big(X^{P,\phi}_{n,j},w^{P,\theta}_{n,j} h_{n,j}^\theta /h_{n,j}^\phi\big)\big\}$ targets $f_{X_{n}|Y_{1:n-1}}(x_{n}|y^*_{1:n-1};\theta)$.
    Lemma \ref{lem:posterior-proper-weights} tells us that $\big\{\big(X^{P,\phi}_{n,j},w^{P,\theta}_{n,j} g^\theta_{n,j}h_{n,j}^\theta /h_{n,j}^\phi \big)\big\}$ therefore targets  $f_{X_{n}|Y_{1:n}}(x_{n}|y^*_{1:n};\theta)$.
    Lemma \ref{lem:change-measure-proper-weights} guarantees that the resampling rule, given by 
    \[
    \big(X^{F,\theta}_{n,j},w^{F,\theta}_{n,j}\big) = \left(X^{P,\phi}_{n,a(j)}, w^{P,\theta}_{n,a(j)} \frac{g^\theta_{n,a(j)}h_{n,a(j)}^\theta}{g^\phi_{n,a(j)}h_{n,a(j)}^\phi}\right),
    \]
    with resampling weights proportional to $g^\phi_{n,j}$, therefore also targets $f_{X_{n}|Y_{1:n}}(x_{n}|y^*_{1:n};\theta)$.
\end{proof}


This has addressed filtering, but not quite yet the likelihood evaluation. For this we use the following lemma.

\begin{lem}[Likelihood Proper Weighting]
    \label{lem:lik-proper-weight}
  $f_{Y_n|Y_{1:n-1}}(y_n^*|y_{1_n-1}^*;\theta)$ is consistently estimated by either the before-resampling estimate,
\begin{equation}\label{L1}
L_n^{B,\theta} =  \frac{\sum_{j=1}^Jg^\theta_{n,j} w^{P,\theta}_{n,j}}{\sum_{j=1}^J  w^{P,\theta}_{n,j}},
\end{equation}
or by the after-resampling estimate,
\begin{equation}\label{L2}
L_n^{A,\theta} = L^\phi_n \frac{\sum_{j=1}^Jw^{F,\theta}_{n,j}}{\sum_{j=1}^J  w^{P,\theta}_{n,j}}.
\end{equation}
where $L^\phi_n$ is as defined in the various algorithms.
\end{lem}

Here, (\ref{L1}) is a direct consequence of our earlier result that $\{ \big(X^{P,\theta}_{n,j},w^{P,\theta}_{n,j}\big) \}$ in the case of MOP and $\{ \big(X^{P,\phi}_{n,j},w^{P,\theta}_{n,j}\big) \}$ in the case of DOP (with a slight abuse of notation here, as the weight expressions are in fact different for both algorithms) targets $f_{X_{n}|Y_{1:n-1}}(x_{n}|y^*_{1:n-1};\theta)$.
To see  (\ref{L2}),
we write the numerator of (\ref{lem:change-measure-proper-weights}) as
\[
L^\phi_n \sum_{j=1}^J \left[ \frac{g^\theta_{n,j}}{g^\phi_{n,j}} w^{P,\theta}_{n,j}\right] \frac{g^\phi_{n,j}}{L_n^\phi}
= L^\phi_n \sum_{j=1}^J w_{n,j}^{FC,\theta} \frac{g^\phi_{n,j}}{L_n^\phi}
\]
Using Lemma \ref{lem:change-measure-proper-weights}, we resample according to probabilities $\frac{g^\phi_{n,j}}{L_n^\phi}$ to see this is properly estimated by
\[
L^\phi_n \sum_{j=1}^J w^{F,\theta}_{n,j},
\]
from which we obtain (\ref{L2}).

Using Lemma \ref{lem:lik-proper-weight}, we obtain a likelihood estimate,
\[
L^{A,\theta} = \prod_{n=1}^N \left( L^\phi_n \, \frac{\sum_{j=1}^J w^{F,\theta}_{n,j}}{\sum_{j=1}^J w^{P,\theta}_{n,j}}\right).
\]
Since $w^{F,\theta}_{n,j}=w^{P,\theta}_{n+1,j}$, this is a telescoping product. The remaining terms are
$\sum_{j=1}^J w^{P,\theta}_{0,j} = J$ on the denominator and $\sum_{j=1}^J w^{F,\theta}_{N,j}$ on the numerator.
This derives the MOP/DOP estimate.

\kevin{IS THIS UNBIASED?}

$L^{B,\theta}$ should generally be preferred in practice, since there is no reason to include the extra variability from resampling when calculating the conditional log likelihood, but it lacks the nice telescoping product that lets us derive exact expressions for the gradient later.



\section{Consistency of Off-Parameter Resampled Gradient Estimates}
\label{appendix:consistency}

We now provide a result showing that under sufficient regularity conditions, one can interchange the order of differentiation and expectation of functionals of off-parameter resampled particle estimates, showing that if an off-parameter resampled particle estimate is consistent for some estimand, its derivative is consistent for the derivative of the estimand as well.

One may wonder why this may be necessary, given that we have already shown strong consistency for measurable bounded functionals in Theorem \ref{thm:off-parameter-targeting}. If we require the gradient to be bounded as well, then the gradient is also a bounded functional. The answer is that the strong consistency is for expectations under the filtering distribution, so Theorem \ref{thm:off-parameter-targeting} only establishes strong consistency of the gradient of the estimate to its expectation under the filtering distribution, $\nabla_\theta \eta_n^J(h_\theta)  \stackrel{a.s.}{\to} \eta_n(\nabla_\theta h_\theta)$, which is not a-priori the gradient of the estimand $\nabla_\theta \eta_n(h_\theta)$. We require an interchange of the derivative and expectation, which we show is possible when the particle estimates have two continuous uniformly bounded derivatives over all $J$ below.

\begin{thm}[Off-Parameter Particle Filters Yield Strongly Consistent Estimates of Derivatives of Functionals]
    Let $h_\theta : \gX \to \R$ be a measurable bounded functional of particles, where $\eta_n^J(h_\theta)$ has two continuous derivatives uniformly bounded over all $J$ by $H^*$ for almost every $\omega\in\Omega$ and $\theta \in \Theta$. If it holds that $\eta_n^J(h_\theta) \stackrel{a.s.}{\to} \eta(h_\theta) = h^*_\theta$, then we also have that $\nabla_\theta \eta_n^J(h_\theta)  \stackrel{a.s.}{\to} \eta_n(\nabla_\theta h_\theta) = \nabla_\theta \eta_n(h_\theta) = \nabla_\theta h^*_\theta$. 
\end{thm}
\begin{proof}
    Fix $\omega \in \Omega$. The sequence $(\nabla_\theta \eta_n^J(h_\theta)(\omega))_{J \in \mathbb{N}}$ is uniformly bounded over all $J$, by assumption. The sequence is also uniformly equicontinuous. To see this, by assumption, the second derivative of $\eta_n^J(h_\theta)(\omega)|_{\theta=\theta'}$ is also uniformly bounded over all $J$ for almost every $\omega\in \Omega$ and every $\theta' \in \Theta$. A set of functions with derivatives bounded by the same constant is uniformly Lipschitz, and therefore uniformly equicontinuous. So the sequence $(\eta_n^J(h_\theta)\omega))_{J \in \mathbb{N}}$ is uniformly equicontinuous over $\theta$ for almost every $\omega \in \Omega$. 
    Explicitly, for almost every $\omega \in \Omega$ and every $\epsilon>0$, there exists some $\delta(\omega)>0$ such that for every $||\theta - \theta'||_{\infty}<\delta$ and every $J \in \mathbb{N}$ we have that
    \begin{equation}
    \big||\eta_n^J(h_\theta)(\omega)-\eta_n^J(h_{\theta'})(\omega)\big||_\infty < \epsilon.
    \end{equation}
    Then, by Arzela-Ascoli, there exists a uniformly convergent subsequence. We claim that there is only one subsequential limit. When the gradient is bounded, we can treat the gradient as a bounded functional. So by Theorem \ref{thm:off-parameter-targeting} the sequence $(\eta_n^J(h_\theta)(\omega))_{J \in \mathbb{N}}$ converges pointwise for $\theta=\phi$ and almost every $\omega \in \Omega$, and there is therefore only one subsequential limit. The sequence therefore converges uniformly to its limit $\lim_{J \to \infty} \eta_n^J(h_\theta)(\omega).$ Therefore, with uniform convergence for the derivatives established, we can swap the limit and derivative, and obtain that for almost every $\omega \in \Omega$, 
    \begin{equation}
    \lim_{J \to \infty} \eta_n^J(h_\theta)(\omega) = \nabla_\theta \lim_{J \to \infty} \eta_n^J(h_\theta)(\omega).
    \end{equation}
Again from Theorem \ref{thm:off-parameter-targeting}, we know that
    \begin{equation}\eta_n^J(h_\theta)(\omega) \to \eta_n(h_\theta) = h^*_\theta\end{equation} for almost every $\omega\in\Omega$.
    We then have that for almost every $\omega \in \Omega$, 
    \begin{equation}
    \lim_{J \to \infty} \eta_n^J(h_\theta)(\omega) = \nabla_\theta \eta_n(h_\theta) = \nabla_\theta h^*_\theta,
    \end{equation}
    as we wanted. 
    \end{proof}
The proof of Theorem \ref{thm:mop-grad-consistency} is now merely a corollary, where we apply $\eta_n^J (h_\theta) = \hat\lik^1_J(\theta)$, $\eta_n(h_\theta) = \lik(\theta) = h_\theta^*$, and then use the continuous mapping theorem. 



% For completeness, we now provide a longer version of the proof of Theorem \ref{thm:mop-grad-consistency}.
% \begin{proof}
%     Fix $\omega \in \Omega$, and set $\phi = \theta$, where $\theta$ is the point we wish to evaluate the gradient at. The sequence $(\nabla_\theta \hat\lik^1_J(\theta)(\omega))_{J \in \mathbb{N}}$ is uniformly bounded over all $J$, by Assumption \ref{assump:local-bounded-derivative}. The sequence is also uniformly equicontinuous. To see this, by Assumption \ref{assump:local-bounded-derivative}, the second derivative of $\hat\lik^1_J(\theta)(\omega)|_{\theta=\theta'}$ is also bounded by $H^*$ for almost every $\omega\in \Omega$ and every $\theta' \in \Theta$. A set of functions with derivatives bounded by the same constant is uniformly Lipschitz, and therefore uniformly equicontinuous. So the sequence $(\nabla_\theta \hat\lik^1_J(\theta)(\omega))_{J \in \mathbb{N}}$ is uniformly equicontinuous for almost every $\omega \in \Omega$. 
%     Explicitly, for almost every $\omega \in \Omega$ and every $\epsilon>0$, there exists some $\delta(\omega)>0$ such that for every $||\theta - \theta'||_{\infty}<\delta$ and every $J \in \mathbb{N}$ we have that
%     \begin{equation}||\nabla_\theta \hat\lik^1_J(\theta)(\omega)-\nabla_\theta \hat\lik^1_J(\theta')(\omega)||_\infty < \epsilon.\end{equation}


%     Then, by Arzela-Ascoli, there exists a uniformly convergent subsequence. We claim that there is only one subsequential limit. We know that when as the gradient is bounded by Assumption \ref{assump:local-bounded-derivative} we can treat the gradient as a bounded functional. So by Lemma \ref{lem:posterior-proper-weights} the sequence $(\nabla_\theta \hat\lik^1_J(\theta)(\omega))_{J \in \mathbb{N}}$ converges pointwise for $\theta=\phi$ and almost every $\omega \in \Omega$, and there is therefore only one subsequential limit. The sequence therefore converges uniformly to its limit $\lim_{J \to \infty} \nabla_\theta \hat\lik^1_J(\theta)(\omega).$ Therefore, with uniform convergence for the derivatives established, we can swap the limit and derivative, and obtain that for almost every $\omega \in \Omega$, 
%     \begin{equation}\lim_{J \to \infty} \nabla_\theta \hat\lik^1_J(\theta)(\omega) = \nabla_\theta \lim_{J \to \infty} \hat\lik^1_J(\theta)(\omega).\end{equation}

%     From Lemma \ref{lem:lik-proper-weight}, we know that
%     \begin{equation}\hat{\lik}^1_J(\theta)(\omega) \to \lik(\theta)\end{equation} for almost every $\omega\in\Omega$.

%     We then have that for almost every $\omega \in \Omega$, 
%     \begin{equation}\lim_{J \to \infty} \nabla_\theta \hat\lik^1_J(\theta)(\omega) = \nabla_\theta \lim_{J \to \infty} \hat\lik^1_J(\theta)(\omega) = \nabla_\theta \lik(\theta),\end{equation}
%     as we wanted. The result then follows by the continuous mapping theorem. 
    
% \end{proof}

\section{Bias-Variance Analysis}
\label{appendix:biasvar}


In this section, we prove various rates on the bias, variance, and MSE of MOP-$\alpha$. 
First, we note the following relation between the Dobrushin contraction coefficient and the alpha-mixing coefficients in our context below.
\begin{lem}
    \label{lem:dobrushin-implies-alpha-mixing}
    Setting $X$ to be the particle collection at time $m$, and $Y$ at time $n$, we have that the alpha mixing coefficients,
    $$\int | f_{XY} - f_X f_Y | < \alpha,$$
    are bounded by the Dobrushin coefficient, i.e we have
    $$\int | f_{Y|X}(y|x_1) - f_{Y|X}(y|x_2) |dy < \alpha$$ for all $x_1$, $x_2$.
\end{lem}

\begin{proof}
    We rewrite the alpha-mixing assertion as 
    $$\int { \int |f_Y|X(y|x) - f_Y(y)| dy } f_X dx.$$
    We claim that the Dobrushin coefficient implies $$\int | f_Y|X(y|x_1) - f_Y(y) |dy < \alpha$$ for all $x_1$. This is shown as follows:
    \begin{align*}
        \int | f_{Y|X}(y|x) - f_Y(y) |dy 
        &= \int \left|  \int [f_{Y|X}(y|x_1) - f_{Y|X}(y|x)] f_X(x)dx \right|dy \\
        &< \int   \int  | f_{Y|X}(y|x_1) - f_{Y|X}(y|x)] | dy f_X(x)dx \\
        &< \int \alpha f_X(x)dx \\
        &=\alpha
    \end{align*}
    We then have the desired result.
\end{proof}

\subsection{Warm Up: MOP-$0$ Variance Bound}
We now prove the variance bound, presented below. 
\begin{thm}
    
\end{thm}
We made Assumptions \ref{assump:bounded-process} and \ref{assump:bounded-measurement} to leverage the results from \cite{karjalainen23} on the forgetting of the particle filter. This is required to show error bounds on the gradient estimates that we provide -- namely that the error of the MOP-$1$ estimator that corresponds to that of \cite{poyiadjis11} is $O(N^2)$, and that the variance of MOP-$\alpha$ for any $\alpha<1$ is $O(N)$. 

We can decompose the MOP-$\alpha$ estimator as follows. When $\theta=\phi$,
$$\hat{\mathcal{L}}(\theta):=\prod_{n=1}^N L_n^{A, \theta, \alpha}=\prod_{n=1}^N L_n^\phi \cdot \frac{\sum_{j=1}^J w_{n, j}^{F, \theta}}{\sum_{j=1}^J w_{n, j}^{P, \theta}}=\prod_{n=1}^N L_n^{A, \theta, \alpha}=\prod_{n=1}^N L_n^\phi \cdot \frac{\sum_{j=1}^J w_{n, j}^{F, \theta}}{\sum_{j=1}^J (w_{n-1, j}^{F, \theta})^\alpha}.$$
Now, to illustrate the proof strategy for the general case in an easier context, we first analyze the special case when $\alpha=0.$ 

\begin{proof}
When $\alpha=0$, we have:
\begin{equation}
    \hat{\lik}(\theta) := \prod_{n=1}^N L_n^{A, \theta, \alpha} = \prod_{n=1}^N L_n^\phi \cdot \frac{\sum_{j=1}^J w_{n,j}^{F,\theta}}{\sum_{j=1}^J w_{n,j}^{P,\theta}} = \prod_{n=1}^N L_n^\phi \cdot \frac{1}{J}\sum_{j=1}^J s_{n,j} = \prod_{n=1}^N L_n^\phi \cdot \frac{1}{J}\sum_{j=1}^J \frac{f_{Y_n|X_n}(y_n^*|x_{n,j}^{P, \theta})}{f_{Y_n|X_n}(y_n^*|x_{n,j}^{P, \phi})}.
\end{equation}
Similarly to the proof of Lemma \ref{lem:mop-0-formula}, its gradient when $\theta=\phi$ is therefore 
\begin{align*}
    \nabla_\theta \hat{\ell}(\theta) &:= \sum_{n=1}^N \nabla_\theta \log\left(L_n^\phi \frac{1}{J} \sum_{j=1}^J s_{n,j}\right) \\
    &= \sum_{n=1}^N \frac{\nabla_\theta \left(L_n^\phi \frac{1}{J} \sum_{j=1}^J s_{n,j}\right)}{\left(L_n^\phi \frac{1}{J} \sum_{j=1}^J s_{n,j}\right)} \\
    &= \sum_{n=1}^N \frac{\sum_{j=1}^J \nabla_\theta s_{n,j}}{\sum_{j=1}^J s_{n,j}} \\
    &= \sum_{n=1}^N \frac{1}{J} \sum_{j=1}^J \frac{\nabla_\theta f_{Y_n|X_{n}}(y_n^*|x_{n,j}^{F, \theta}; \theta)}{f_{Y_n|X_{n}}(y_n^*|x_{n,j}^{F, \phi}; \phi)} \\
    &= \frac{1}{J} \sum_{n=1}^N \sum_{j=1}^J \nabla_\theta \log\left(f_{Y_n|X_{n}}(y_n^*|x_{n,j}^{F, \theta}; \theta)\right),
\end{align*}
where we use the log-derivative trick, observe that $\sum_{j=1}^J s_{n,j} = J$ when $\theta=\phi$, and use the log-derivative trick where $\theta=\phi$ again. The point here was that we establish that the gradient of the log-likelihood estimate is given by the sum of terms over all $N$ and $J$:
$$\nabla_\theta \hat{\ell}(\theta) = \frac{1}{J} \sum_{n=1}^N \sum_{j=1}^J \nabla_\theta \log\left(f_{Y_n|X_{n}}(y_n^*|x_{n,j}^{F, \theta}; \theta)\right).$$
Therefore,
\begin{align*}
    \Var(\nabla_\theta \hat\ell(\theta)) &= \frac{1}{J^2}\Var\left(\sum_{n=1}^N\sum_{j=1}^{J}\nabla_\theta \log\left(f_{Y_n|X_{n}}(y_n^*|x_{n,j}^{F, \theta}; \theta)\right)\right) \\
    &= \frac{1}{J^2}\sum_{n=1}^N\Var\left(\sum_{j=1}^{J}\nabla_\theta \log\left(f_{Y_n|X_{n}}(y_n^*|x_{n,j}^{F, \theta}; \theta)\right)\right) \\
    &+ 2\sum_{m<n}\Cov\left(\frac{1}{J}\sum_{j=1}^{J}\nabla_\theta \log\left(f_{Y_m|X_{m}}(y_m^*|x_{m,j}^{F, \theta}; \theta)\right), \frac{1}{J}\sum_{j=1}^{J}\nabla_\theta \log\left(f_{Y_n|X_{n}}(y_n^*|x_{n,j}^{F, \theta}; \theta)\right)\right) \\
    &= \sum_{n=1}^N \Var(\nabla_\theta\hat\ell_n(\theta)) + 2\sum_{m<n} \Cov(\nabla_\theta\hat\ell_m(\theta),\nabla_\theta\hat\ell_n(\theta)).
\end{align*}
Here, we use Assumptions \ref{assump:bounded-process} and \ref{assump:bounded-measurement} that ensure strong mixing. We know from Theorem 3 of \cite{karjalainen23} that when $\textbf{M}_{n,n+k}$ is the $k$-step Markov operator from timestep $n$ and $\beta_{\text{TV}}(M) = \sup _{x, y \in E}\|M(x, \cdot)-M(y, \cdot)\|_{\mathrm{TV}}=\sup _{\mu, \nu \in \mathcal{P}, \mu \neq \nu} \frac{\|\mu M-\nu M\|_{\mathrm{TV}}}{\|\mu-\nu\|_{\mathrm{TV}}}$ is the Dobrushin contraction coefficient of a Markov operator, 
$$\beta_{\text{TV}}(\textbf{M}_{n,n+k}) \leq (1-\epsilon)^{\floor{k/(c\log(J))}},$$
i.e. the mixing time of the particle filter is $O(\log(J))$, where $\epsilon$ and $c$ depend on $\bar{M}, \underbar{M}, \bar{G}, \underbar{G}$ in \ref{assump:bounded-process} and \ref{assump:bounded-measurement}. 

%\kevin{How does the dobrushin contraction coefficient relate to the alpha-mixing coefficients? If you have a bound on the mixing time $b$, then the alpha mixing coefficients should just be $\alpha(t) \leq e^{-t/b}$.}

By Lemma \ref{lem:dobrushin-implies-alpha-mixing}, the particle filter itself is strong mixing, with $\alpha$-mixing coefficients $\alpha(k) \leq (1-\epsilon)^{\floor{k/(c\log(J))}}$. Therefore, functions of particles are strongly mixing as well, with $\alpha$-mixing coefficients bounded by the original (to see this, observe that the $\sigma$-algebra of the functionals is contained within the original $\sigma$-algebra). Therefore, by Davydov's inequality, noting that $\nabla_\theta\hat\ell_n(\theta)\leq G'(\theta)$ by Assumption \ref{assump:bounded-measurement}, and WLOG assuming $\E[(\nabla_\theta\hat\ell_m(\theta))^4]^{1/4}\leq\E[(\nabla_\theta\hat\ell_n(\theta))^4]^{1/4}$, we see that
\begin{align*}
    \Cov(\nabla_\theta\hat\ell_m(\theta), \nabla_\theta\hat\ell_n(\theta)) 
    &\leq \alpha(n-m)^{1/2}\E[(\nabla_\theta\hat\ell_m(\theta))^4]^{1/4}\E[(\nabla_\theta\hat\ell_n(\theta))^4]^{1/4}\\
    &\leq \alpha(n-m)^{1/2}\E[(\nabla_\theta\hat\ell_n(\theta))^4]^{1/2}.
\end{align*}
To bound this, we use the fact that $$\E[(\nabla_\theta\hat\ell_n(\theta))^4] = \E[(\nabla_\theta\hat\ell_n(\theta)-\E[\nabla_\theta\hat\ell_n(\theta)])^4]+\E[(\nabla_\theta\hat\ell_n(\theta))^2]^2$$
alongside Lemma 2 of \cite{karjalainen23}, which shows that
$$\E[(\nabla_\theta\hat\ell_n(\theta)-\E[\nabla_\theta\hat\ell_n(\theta)])^4] \lesssim \frac{G'(\theta)^4}{J^2}, \;\; \E[(\nabla_\theta\hat\ell_n(\theta)-\E[\nabla_\theta\hat\ell_n(\theta)])^2] \lesssim \frac{G'(\theta)^2}{J}.$$
It follows that 
$$\E[(\nabla_\theta\hat\ell_n(\theta))^4]^{1/2} \lesssim  \sqrt{\frac{G'(\theta)^4}{J^2}+\left(\frac{G'(\theta)^2}{J}\right)^2} = \frac{G'(\theta)^2}{J},$$
and we conclude that 
$$\Cov(\nabla_\theta\hat\ell_m(\theta), \nabla_\theta\hat\ell_n(\theta)) \leq (1-\epsilon)^{\frac{1}{2}\floor{\frac{|n-m|}{c\log(J)}}}\frac{G'(\theta)^2}{J}.$$
Putting it all together, we see that
\begin{align*}
    \Var(\nabla_\theta \hat\ell(\theta)) &= \sum_{n=1}^N \Var(\nabla_\theta\hat\ell_n(\theta)) + 2\sum_{m<n} \Cov(\nabla_\theta\hat\ell_m(\theta),\nabla_\theta\hat\ell_n(\theta))\\
    &\leq \frac{NG'(\theta)^2}{J} + 2\sum_{n=1}^{N-1} (N-n)(1-\epsilon)^{\frac{1}{2}\floor{\frac{n}{c\log(J)}}}\frac{G'(\theta)^2}{J} \\
    &\leq \frac{G'(\theta)^2}{J} \left(N + 2\sum_{n=1}^{N-1} (N-n) (1-\epsilon)^{\frac{1}{2}\floor{\frac{n}{c\log(J)}}}\right) \\
    &\lesssim \frac{G'(\theta)^2N}{J}.
\end{align*}
\end{proof}


\paragraph{Remark:} Note that the factor of $N$ that pops up here is due to the use of the unnormalized gradient. If one divides the gradient estimate by $\sqrt{N}$ before usage, the variance does not depend on the horizon. If one divides by $N$, the error is $O(1/\sqrt{NJ})$.

\section{Figures for Bayesian Inference}
\label{appendix:bayes}

\begin{figure}[H]
    \centering
    \includegraphics[width=\textwidth/\real{1.25}]{imgs/pmcmc/mh.png}
    \caption{Convergence diagnostics for the Metropolis-Hastings variant particle MCMC with the random walk proposal and an informative empirical prior from IF2. Here, we display the results for the trend parameter in the Dhaka cholera model of \cite{king08}.}
    \label{fig:mh}
\end{figure}
\FloatBarrier
\newpage

\begin{figure}[H]
    \centering
    \includegraphics[width=\textwidth/\real{1.25}]{imgs/pmcmc/nuts.png}
    \caption{Convergence diagnostics for a No-U-Turn Sampler (NUTS) with uniform priors on a compact set. Again, we display the results for the trend parameter in the Dhaka cholera model of \cite{king08}. The NUTS sampler explores more of the posterior than the Metropolis-Hastings sampler, but fails to converge quickly.}
    \label{fig:nuts}
\end{figure}
\FloatBarrier
\newpage

\begin{figure}[H]
    \centering\
    \includegraphics[scale=0.30]{imgs/pmcmc/nuts_eb/Recovery_Rate.png}
    \includegraphics[scale=0.30]{imgs/pmcmc/nuts_eb/Death_Rate.png}
    \includegraphics[scale=0.30]{imgs/pmcmc/nuts_eb/Reciprocal_of_Mean_Immunity_Duration.png}
    \includegraphics[scale=0.30]{imgs/pmcmc/nuts_eb/Trend_in_Force_of_Infection.png}
    \includegraphics[scale=0.30]{imgs/pmcmc/nuts_eb/Noise_in_Force_of_Infection.png}
    \includegraphics[scale=0.30]{imgs/pmcmc/nuts_eb/Measurement_Noise.png}
    \caption{Posterior estimates from NUTS powered by MOP-$\alpha$ with the informative nonparametric empirical prior obtained from a kernel density estimator on the IF2 parameter cloud. The posterior estimates from each chain are largely in agreement.}
    \label{fig:posteriors}
\end{figure}

\FloatBarrier
\newpage

%%% Each figure should be on its own page

\bibliographystyle{apalike}
\bibliography{paper/bib-ifad}


\end{document}