\documentclass[11pt]{article} 

\begin{document}

\noindent Dear PNAS editors,
\newline

Please consider our manuscript, ``Automatic Differentiation Accelerates Inference for Partially Observed Markov Processes", for publication in PNAS.

The contribution of this paper has widespread implications due to the diverse uses of this model class in ecology, engineering, epidemiology, finance, and elsewhere. Our method promises substantially improved inference wherever simulation-based inference is used for the full-information likelihood in complex partially observed Markov processes. That is why we believe that publication of our article in PNAS would facilitate a broad range of scientific advances. 

Although our algorithm can be implemented by a competent programmer familiar with current machine learning tools, the theory and motivation behind our contribution is rather technical. The justification for our new algorithm requires deep familiarity with the theory of sequential Monte Carlo (SMC) and some experience with the theory and practice of automatic differentiation (AD). Thus, relatively few researchers are well placed to fully appreciate and evaluate our contribution. Leading SMC researchers who have already considered algorithms incorporating AD include Arnaud Doucet, Frank Wood, and George Deligiannidis. Nicolas Chopin is a leading SMC theorist who is also involved in developing SMC computation so he would also be a qualified referee. It would assist the evaluation of our contribution if at least one referee or invited editor could be drawn from this select group of experts.

This manuscript focuses on likelihood-based non-Bayesian inference. We also demonstrate relevance to Bayesian inference, but that is not our primary topic. Therefore, we ask for at least one referee who has experience working outside the Bayesian paradigm.

\vspace{3ex}

\noindent Sincerely,

\noindent Kevin Tan, Ed Ionides and Giles Hooker
\end{document}
